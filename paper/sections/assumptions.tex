\section{Assumptions}\label{sec:introduction}

the author determined that unit tests could be
created programmatically by assuming that all code executing under the
functional test was in fact correct, then using the inputs and outputs from each
function called during the functional test as the inputs and outputs for the
unit tests.  

The author acknowleges that software engineers may not 
wish to assume that a apparently working functional test indicates accuracy of 
all internally executed components. Regardless, the creation of such unit 
tests would still provide a syntactically accurate unit
test file for the software engineer to start from, rather than create each unit
test from scratch.  

In addition to saving time from creating the test
boilerplate, the true expense saved is that of manually defining the desired
inputs and correct outputs.  This author believes there is value in this
approach even if the generated tests must still be verified for accuracy, as it
could be faster to manually inspect code for accuracy than develop and verify
such code from scratch.

Another assumption required for successful execution is that the \textit{repr()} method
of each object generates valid Python code than can be used to re-created that
object, or that such a function can be developed and used to temporarily
overwrite a \textit{repr\(\)} method not meeting this requirement.  The author demonstrates
this in the repository code by overwriting the \textit{repr\(\)} method of the
Pandas DataFrame class in Listing 1.

 
 \lstinputlisting[%
 language=Python,%
 numbers=left,
 caption={Overwriting incompatible \textit{repr} method},%
 label={lst:pandas_repr},%
]{examples/pandas\_repr.py}
