\section{Test Generation with the decorator}\label{sec:approach}

%%
%% General information about The decorator
%%
The decorator is written in Python and requires at least Python~3.9 to run.
%
The decorator was developed and tested on a Windows 10 PC but in theory
should run in *nix environments.
%
The author releases it under the GNU LGPL open-source licence.
%

%%%%%%%%%%%%%%%%%%%%%%%%%%%%%%%%%%%%%%%%%%%%%%%%%%%%%%%%%%%%%%%%%%%%%%%%%%%%%%%%

\subsection{The decorator's Components}\label{sec:approach-internal}

%%
%% Reference the figure
%%
\Cref{fig:The decorator-components} shows the components of The decorator 
and their interactions
throughout the test-generation process.
%

% Show the decorator syntax and what it actually looks like under the hood,
% noting that f, args, and kwargs are accessible

% Next, discss the need to access global state (read/write).

% Note that the inspect module provides thie option via dumping bytecode for f

% Note that we have to track test coverage manually, or at least provide a unique
% name for the coverage database

% Provide snippets of the key classes and logic if not already shown above




% vim: spelllang=en_GB
