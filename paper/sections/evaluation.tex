\section{Tester, Test Thyself}\label{sec:evaluation}

The decorator was evaluated informally with three separate approaches.
The author began by crafting simple but demonstrative test cases 
(found in the \textit{tests/} folder in the associated repository.) Next,
the author applied the concept of programmatically generated unit tests
to the code itself with positive results.  Finally, as previously mentioned, 
the author applied his work to another project involving parsing C code. 
These three approaches are described detail below:

\subsection{Manually Created Tests}\label{sec:eval-1}
\hfill\\
The author created a variety of tests to ensure the unit test generation 
code functioned properly. The Procedural Division example is explained in depth
below and the others are briefly summarized.

The \textit{tests/example\_procedural\_division} test was intended 
to test procedural (as opposed to object-oriented or functional) code that:  

\begin{enumerate}
    \item Returned a string given two ints
    \item Wrote to a global variable
    \item Raised two different types of exceptions
\end{enumerate}

The author wrote a \textit{divide\_ints} function as follows:

\lstinputlisting[%
  language=Python,%
  numbers=left,
  caption={divide\_ints.py},%
  label={lst:Divide Ints},%
]{examples/divide\_ints\_1.py}

Then wrote an ad-hoc "test" without assertions, but calls 
\textit{divide\_ints} with a variety of inputs, as shown below:

\lstinputlisting[%
  language=Python,%
  numbers=left,
  caption={test divide\_ints()},%
  label={lst:Test Divide Ints},%
]{examples/divide\_ints\_2.py}

Running this code like so:

\begin{lstlisting}[language=bash]
    $ python divide_ints.py
\end{lstlisting}
    
Produces the following unit test (modified only slightly below to reduce line
 breaks):

\lstinputlisting[%
  language=Python,%
  numbers=left,
  caption={test divide\_ints()},%
  label={lst:Test Divide Ints},%
]{examples/divide\_ints\_3.py}

Executing the unit test is simple\footnote{Due to the verbose way pytest prints the 
all the parameters of  parameterized tests such as these, the author removed
them for the sake simpler display.  The reader is encouraged to run the code 
as shown on their own machine}:

\begin{lstlisting}[language=bash]
pytest -s -v test_divide_ints.py
========== test session starts ==========
platform win32 -- Python 3.11.7, pytest-7.4.4, pluggy-1.4.0 -- <@\textcolor{orange}{PATH REDACTED}@>\.venv\Scripts\python.exe
cachedir: .pytest_cache
rootdir: <@\textcolor{orange}{PATH REDACTED}@>\tests\example_procedural_division
plugins: cov-5.0.0
collected 4 items

test_divide_ints.py::test_divide_ints[<@\textcolor{orange}{Test \#1 arguments SNIPPED}@>] <@\textcolor{green}{PASSED}@>
test_divide_ints.py::test_divide_ints[<@\textcolor{orange}{Test \#2 arguments SNIPPED}@>] <@\textcolor{green}{PASSED}@>
test_divide_ints.py::test_divide_ints[<@\textcolor{orange}{Test \#3 arguments SNIPPED}@>] <@\textcolor{green}{PASSED}@>
test_divide_ints.py::test_divide_ints[<@\textcolor{orange}{Test \#4 arguments SNIPPED}@>] <@\textcolor{green}{PASSED}@>
\end{lstlisting}

The reader is encouraged to study the other examples in the repository.
All examples can be executed via the 
\textit{tests/test\_all.*} scripts.  Those
scripts can easily be used to determine the invocation required to run each 
test individually.  They are summarized very briefly below:

The \textit{tests/example\_fizzbuzz} test is nearly identical to the 
test described above: tests modification of a global value.

The \textit{tests/example\_all\_types} test was designed to test
the code against a variety of built-in Python types, 
such as ints, strings, sets, lists, tuples, and dictionaries.

The \textit{tests/example\_pass\_by\_assignment} test was designed to test
functions that modify mutable arguments such as lists. Not a comprehensive test,
but demonstrates that the concept works on one such example.

The \textit{tests/example\_oo\_car} test was designed to test
object-oriened code. Not complete at the time of submission, but worked in a 
previous branch prior to adding support for the test above.
The author is confident that this feature will be working by 
fall of 2024.


%
\subsection{Application of this code to itself}\label{sec:eval-2}
\hfill\\
The author naturally sought to determine the effectiveness of automatic
test code generation against the very code doing just that.  Although not 
every function in the referenced code was decorated for automatic testing, 
as of this writing five functions in 
\textit{unit\_test\_generator\_decorator.py} were successfully decorated, 
resulting in automatically generated, successfully passing unit tests for this
very code.  Using the previous Procedural Division example (and others) 
the reader should see the following test files created in addition to 
\textit{test\_divide\_ints.py}

\begin{enumerate}
    \item test\_coverage\_str\_helper.py
    \item test\_meta\_program\_function\_call.py
    \item test\_normalize\_args.py
    \item test\_update\_global.py
    \item test\_update\_metadata.py
\end{enumerate}

These tests also pass when run with this command in \textit{tests/example\_procedural\_division/}:
\begin{lstlisting}[language=bash]
    pytest -s -v .
\end{lstlisting}

\subsection{Application of this code to an external project}\label{sec:eval-2}
As previously mentioned and put simply, this project was applied to a C code parser, 
specifically to a Python function designed to parse C code line by line
and extract the name of any functions on each line.  The author applied this
decorator to the parsing function and passed a C file to the parser function.
%
Due to the design of the parser, some lines were parsed multiple times.  
However, the decorator prevented duplicate results as previously discussed, 
early instantly producing over three dozen unique test cases 
from a single functional test and revealing two bugs to the author.
%
As noted, some work remains, such as completing the decorator and helper logic
to handle object-oriented code.  However, due to the variety of successful 
tests noted above, the author maintains an optimistic outlook that this concept
of generating unit tests from functional tests can be a valuable paradigm.

% vim: spelllang=en_GB
