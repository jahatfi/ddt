\section{Tester, Test Thyself}\label{sec:evaluation}

The decorator was evaluated informally with three separate approaches.
The author began by crafting simple but demonstrative test cases 
(found in the \textit{tests/} folder in the associated repository.) Next,
the author applied the concept of programmatically generated unit tests
to the code itself with positive results. With two key logical checks it
was possible to apply the decorator to its very own support code.  Those
checks were to:

\begin{enumerate}
  \item Immediately return if the pytest module was loaded, indicating a unit (not integration) test is currently being executed
  \item Immediately return if the call stack contained a loop, e.g. A$\rightarrow$B$\rightarrow$A
\end{enumerate}

These checks are shown in Listing~\ref{lst:Limit Recursion}.

\lstinputlisting[%
  language=Python,%
  numbers=left,
  caption={limit\_recursion.py},%
  label={lst:Limit Recursion},%
]{examples/limit_recursion.py}

the carefule examination of the call stack was required to limit 
recursive decorators to just one level. Finally, as previously mentioned, 
the author applied his work to another project involving parsing C code. 
These three approaches are described detail below:

\subsection{Manually Created Tests}\label{sec:eval-1}
The author created a variety of tests to ensure the unit test generation 
code functioned properly. The Procedural Division example is explained in depth
below and the others are briefly summarized.

The \textit{tests/example\_procedural\_division} test was intended 
to test procedural (as opposed to object-oriented or functional) code that:  

\begin{enumerate}
    \item Returned a string given two ints
    \item Wrote to a global variable
    \item Raised two different types of exceptions
\end{enumerate}

The author wrote a \textit{divide\_ints} function as shown 
in Listing~\ref{lst:Divide Ints}.

\lstinputlisting[%
  language=Python,%
  numbers=left,
  caption={divide\_ints.py},%
  label={lst:Divide Ints},%
]{examples/divide\_ints\_1.py}

Then wrote an ad-hoc "test" that calls 
\textit{divide\_ints} with a variety of inputs, but no assertions
as shown in List~\ref{lst:Test Divide Ints}.  Note the useof erroneous inputs
to cover the error handling code.

\lstinputlisting[%
  language=Python,%
  numbers=left,
  caption={test divide\_ints()},%
  label={lst:Test Divide Ints},%
]{examples/divide\_ints\_2.py}

Running this code with pytest as displayed in Listing 6%~\ref{lst:rundivide}.

\begin{lstlisting}[
  language=bash, 
  caption={Executing example to create unit tezst},
  label={lst:rundivide}]
    $ python divide_ints.py
\end{lstlisting}
    
produces the following \textit{test\_divide\_ints.py} containing the following 
unit test (modified only slightly below to reduce line breaks):

\lstinputlisting[%
  language=Python,%
  numbers=left,
  caption={test divide\_ints},%
  label={lst:Test Divide Ints},%
]{examples/divide\_ints\_3.py}

Lines 18-20 set up the parameterization decorator, defining the inputs to each test,
and the inputs to each are defined in a list of tuples spanning lines 23-33, 
34-44, 45-57.  The actual test function follows starting on line 60, monkeypatches
the required ERROR\_CODE global variable on lines 75-76, tests for expected 
exceptions on lines 77-80, and only calls the function on line 82 if it expects 
no exception. The result is verified on lines 82-83
\footnote{due to the complexity of converting from Python object to string
(and sometimes back again), the author was forced due to time contraints to
rely on the \textit{eval} function to convert strings to valid Python.  
The author acknowleges this is a bad practice and hopes to fix it in
future revisions.}. An unexpected exception (or incorrect 
(exception, exception message) 
pair) would cause the unit test to fail.  Finally, global variables expected to be modified 
are checked for correct values on lines 85-92.  Note that not all test cases 
were retained due to the deduplication logic discussed in Section~\ref{sec:tuning-1}.

Executing the unit test is simple as shown in Listing~\ref{Running one of the generated unit tests}\footnote{Due to the verbose way pytest prints the 
all the parameters of  parameterized tests such as these, the author removed
them for the sake of simpler display.  The reader is encouraged to run the code 
as shown on their own machine}.

\begin{lstlisting}[language=bash, numbers=left, caption={Running one of the generated unit tests}]
pytest -s -v test_divide_ints.py
========== test session starts ==========
platform win32 -- Python 3.11.7, pytest-7.4.4, pluggy-1.4.0 -- <@\textcolor{orange}{PATH REDACTED}@>\.venv\Scripts\python.exe
cachedir: .pytest_cache
rootdir: <@\textcolor{orange}{PATH REDACTED}@>\tests\example_procedural_division
plugins: cov-5.0.0
collected 4 items

test_divide_ints.py::test_divide_ints[<@\textcolor{orange}{Test \#1 arguments SNIPPED}@>] <@\textcolor{green}{PASSED}@>
test_divide_ints.py::test_divide_ints[<@\textcolor{orange}{Test \#2 arguments SNIPPED}@>] <@\textcolor{green}{PASSED}@>
test_divide_ints.py::test_divide_ints[<@\textcolor{orange}{Test \#3 arguments SNIPPED}@>] <@\textcolor{green}{PASSED}@>
test_divide_ints.py::test_divide_ints[<@\textcolor{orange}{Test \#4 arguments SNIPPED}@>] <@\textcolor{green}{PASSED}@>
\end{lstlisting}

The reader is encouraged to study the other examples in the repository.
All examples can be executed via the 
\textit{tests/test\_all.*} scripts.  The reader can inspect those scripts
to determine how to run each test individually. 
They are summarized very briefly below:

The \textit{tests/example\_fizzbuzz} test is nearly identical to the 
test described above: tests modification of a global value.

The \textit{tests/example\_all\_types} test was designed to test
the code against a variety of built-in Python types, 
such as ints, strings, sets, lists, tuples, and dictionaries.

The \textit{tests/example\_pass\_by\_assignment} test was designed to test
functions that modify mutable arguments such as lists. Not a comprehensive test,
but demonstrates that the concept works on one such example.

The \textit{tests/example\_oo\_car} test was designed to test
object-oriented code. Not complete at the time of submission, but worked in a 
previous branch prior to adding support for the test above.
The author hopes to restore this feature to full functionality in future work.
%
\subsection{Application of this code to itself}\label{sec:eval-2}

The author naturally sought to determine the effectiveness of automatic
test code generation against itself.  Although not 
every function in the referenced code was decorated for automatic testing, 
as of this writing five functions in 
\textit{unit\_test\_generator\_decorator.py} were successfully decorated
\footnote{with the exception of a few OOP relate tests, see above.}, 
resulting in automatically generated, successfully passing unit tests for this
very code.  Using the previous Procedural Division example (and others) 
the reader should see the following test files created in addition to 
\textit{test\_divide\_ints.py}

\begin{enumerate}
    \item \textit{test\_coverage\_str\_helper.py}
    \item \textit{test\_meta\_program\_function\_call.py}
    \item \textit{test\_normalize\_args.py}
    \item \textit{test\_update\_global.py}
    \item \textit{test\_update\_metadata.py}
\end{enumerate}

These tests also pass when run with this command in 
\linebreak
\textit{tests/example\_procedural\_division/} directory:
\begin{lstlisting}[language=bash, caption={Running all generated unit
   tests for the division example}]
    pytest -s -v .
\end{lstlisting}

\subsection{Application of this code to an external project}\label{sec:eval-2}
To validate this work on an external codebase, the author applied 
this project to a single function (due to time constraints) of his C code
parser that inspired it.  The author applied this
decorator to a Python function designed to extract the name of any C functions 
from a provided line of code, then ran the sole function test that called 
the decorated function extensively with various and redundant inputs.
The applied decorator worked as desired, including deduplicating results as
previously discussed, rapidly capturing over three dozen unique test cases.
Upon visually inspecting the inputs and outputs, the author 
nearly instantly identified two bugs in the decorated function.
%
After patching both bugs revealed by the unit tests, he re-ran the function test, 
and inspected the newly created unit tests. At that point he was confident that
the generated unit test and dozens of tests cases sufficed for unit testing apart 
from the functional test.

At that point, he removed the decorator, restoring the functional test to its 
original runtime. Thus, under ideal situations, the tester only pays the 
decoration penalty as few as two times per function.  Of course, multiple 
functions can be decorated at once.
%
% vim: spelllang=en_GB
