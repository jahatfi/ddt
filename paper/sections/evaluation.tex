\section{Evaluation}\label{sec:evaluation}

We evaluated \pynguin
by conducting a small experiment for this paper,
using the aforementioned algorithms for test generation.
%
We used \numModules modules from our previous work~\cite{LKF21}
for our evaluation
and ran \pynguin in version~0.17.0~\cite{Luk22Pynguin0170}
\numIterations times on each module
and in each configuration
to minimise the influence of randomness.
%
For the experiment,
we set the timeout for test generation to \timeout.% \footnote{%
%   We provide an artefact containing the raw data %
%   and the evaluation scripts from %
%   \url{https://zenodo.org/TOBEDONE}.%
% }
%
In this work we only give few insights;
%
for a more extensive evaluation
we refer the reader to our previous work~\cite{LKF20,LKF21},
which not only studies the differences
between various algorithms in greater detail
but also investigates on the influence of type information.
%

\begin{figure}[t]
  \centering
  \includegraphics[width=0.95\linewidth]{coverage-over-time}
  \Description{A line plot showing the coverage results over time for the
    six algorithms.
    DynaMOSA performs best, at around \avgCoverage branches covered,
    followed by MOSA, Whole Suite with Archive, MIO, and Whole Suite.
    The Random algorithm yields the lowest coverage results.
  }
  \caption{\label{fig:coverage-over-time}Development of the coverage over time.}
\end{figure}

To gain insights on the performance of the different algorithms,
we measured branch coverage.
%
\Cref{fig:coverage-over-time} shows the development
of the mean coverage per configuration
over the generation time of \timeout.
%
One can clearly see that the search-based techniques
outperform the random algorithm.
%
The five search-based algorithms,
however,
only show small differences,
with DynaMOSA achieving the highest
and Whole Suite the lowest coverage values~(mean branch coverage for
DynaMOSA\@: \avgCoverageDynaMOSA,
MIO\@: \avgCoverageMIO,
MOSA\@: \avgCoverageMOSA,
Random\@: \avgCoverageRandom,
Whole Suite\@: \avgCoverageWS,
and for Whole Suite with archive\@: \avgCoverageWSA).
%
These results are in line with previous research~\cite{CAF+18,PKT18a}
in the context of statically typed languages:
%
the search-based algorithms achieve a higher average coverage
than the random algorithm.
%
Furthermore, DynaMOSA yields the highest coverage values.
%

% vim: spelllang=en_GB
