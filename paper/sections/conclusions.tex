\section{Conclusions}\label{sec:conclusions}

The increasing popularity of Python
requires the availability of many types of tools
to aid the developers.
%
We introduce \pynguin,
an automated unit test generation framework,
designed to support developers
when implementing unit tests manually.
%
\Pynguin is available as a command-line application,
which is the de-facto standard for many developer-aiding tools.
%
While this provides great flexibility for users,
\pynguin's modular design also enables further extensions
and research in the field of automated unit test generation
for dynamically typed languages.
%

%
In this work we have summarised the features of \pynguin.
%
By providing \pynguin as open source,
we hope to foster further research
as well as its applicability in practice.
%
Further information on \pynguin,
its documentation,
and source code
are available at
%
\begin{center}
  \url{https://www.pynguin.eu}
\end{center}


% vim: spelllang=en_GB
