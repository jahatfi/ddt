\documentclass[10pt, conference]{IEEEtran}
%DIF LATEXDIFF DIFFERENCE FILE
%DIF DEL old.tex   Mon Sep 16 01:16:25 2024
%DIF ADD new.tex   Mon Sep 16 01:16:37 2024
\IEEEoverridecommandlockouts


\usepackage{caption}
\usepackage{enumitem}
\usepackage{array}
\usepackage{multirow}
\usepackage{listings}
\lstset{escapeinside={<@}{@>}}
\usepackage{amsmath}
\usepackage{xcolor}
\usepackage{cite}
\usepackage{amsmath,amssymb,amsfonts}
\usepackage{algorithmic}
\usepackage{graphicx}
\usepackage{textcomp}
\usepackage{xcolor}

\definecolor{codegreen}{rgb}{0,0.6,0}
\definecolor{codegray}{rgb}{0.5,0.5,0.5}
\definecolor{codepurple}{rgb}{0.58,0,0.82}
\definecolor{backcolour}{rgb}{0.95,0.95,0.92}

\lstdefinestyle{pythonstyle}{
commentstyle=\color{codegreen},
    keywordstyle=\color{blue},
    numberstyle=\tiny\color{codegray},
    stringstyle=\color{orange},
    basicstyle=\ttfamily\footnotesize,
    breakatwhitespace=false,
    breaklines=true,
    captionpos=b,
    frame=single,
    keepspaces=true,
    numbers=left,
    numbersep=5pt,
    showspaces=false,
    showstringspaces=false,
    showtabs=false,
    tabsize=2
}

\lstdefinestyle{batstyle}{
commentstyle=\color{black},
    keywordstyle=\color{black},
    numberstyle=\color{black},
    stringstyle=\color{black},
    basicstyle=\ttfamily\footnotesize,
    breakatwhitespace=false,
    breaklines=true,
    captionpos=b,
    frame=single,
    keepspaces=true,
    numbers=left,
    numbersep=5pt,
    showspaces=false,
    showstringspaces=false,
    showtabs=false,
    tabsize=2
}

\lstset{style=pythonstyle}


\def\BibTeX{{\rm B\kern-.05em{\sc i\kern-.025em b}\kern-.08em
    T\kern-.1667em\lower.7ex\hbox{E}\kern-.125emX}}
%DIF PREAMBLE EXTENSION ADDED BY LATEXDIFF
%DIF UNDERLINE PREAMBLE %DIF PREAMBLE
\RequirePackage[normalem]{ulem} %DIF PREAMBLE
\RequirePackage{color}\definecolor{RED}{rgb}{1,0,0}\definecolor{BLUE}{rgb}{0,0,1} %DIF PREAMBLE
\providecommand{\DIFadd}[1]{{\protect\color{blue}\uwave{#1}}} %DIF PREAMBLE
\providecommand{\DIFdel}[1]{{\protect\color{red}\sout{#1}}} %DIF PREAMBLE
%DIF SAFE PREAMBLE %DIF PREAMBLE
\providecommand{\DIFaddbegin}{} %DIF PREAMBLE
\providecommand{\DIFaddend}{} %DIF PREAMBLE
\providecommand{\DIFdelbegin}{} %DIF PREAMBLE
\providecommand{\DIFdelend}{} %DIF PREAMBLE
\providecommand{\DIFmodbegin}{} %DIF PREAMBLE
\providecommand{\DIFmodend}{} %DIF PREAMBLE
%DIF FLOATSAFE PREAMBLE %DIF PREAMBLE
\providecommand{\DIFaddFL}[1]{\DIFadd{#1}} %DIF PREAMBLE
\providecommand{\DIFdelFL}[1]{\DIFdel{#1}} %DIF PREAMBLE
\providecommand{\DIFaddbeginFL}{} %DIF PREAMBLE
\providecommand{\DIFaddendFL}{} %DIF PREAMBLE
\providecommand{\DIFdelbeginFL}{} %DIF PREAMBLE
\providecommand{\DIFdelendFL}{} %DIF PREAMBLE
\newcommand{\DIFscaledelfig}{0.5}
%DIF HIGHLIGHTGRAPHICS PREAMBLE %DIF PREAMBLE
\RequirePackage{settobox} %DIF PREAMBLE
\RequirePackage{letltxmacro} %DIF PREAMBLE
\newsavebox{\DIFdelgraphicsbox} %DIF PREAMBLE
\newlength{\DIFdelgraphicswidth} %DIF PREAMBLE
\newlength{\DIFdelgraphicsheight} %DIF PREAMBLE
% store original definition of \includegraphics %DIF PREAMBLE
\LetLtxMacro{\DIFOincludegraphics}{\includegraphics} %DIF PREAMBLE
\newcommand{\DIFaddincludegraphics}[2][]{{\color{blue}\fbox{\DIFOincludegraphics[#1]{#2}}}} %DIF PREAMBLE
\newcommand{\DIFdelincludegraphics}[2][]{% %DIF PREAMBLE
\sbox{\DIFdelgraphicsbox}{\DIFOincludegraphics[#1]{#2}}% %DIF PREAMBLE
\settoboxwidth{\DIFdelgraphicswidth}{\DIFdelgraphicsbox} %DIF PREAMBLE
\settoboxtotalheight{\DIFdelgraphicsheight}{\DIFdelgraphicsbox} %DIF PREAMBLE
\scalebox{\DIFscaledelfig}{% %DIF PREAMBLE
\parbox[b]{\DIFdelgraphicswidth}{\usebox{\DIFdelgraphicsbox}\\[-\baselineskip] \rule{\DIFdelgraphicswidth}{0em}}\llap{\resizebox{\DIFdelgraphicswidth}{\DIFdelgraphicsheight}{% %DIF PREAMBLE
\setlength{\unitlength}{\DIFdelgraphicswidth}% %DIF PREAMBLE
\begin{picture}(1,1)% %DIF PREAMBLE
\thicklines\linethickness{2pt} %DIF PREAMBLE
{\color[rgb]{1,0,0}\put(0,0){\framebox(1,1){}}}% %DIF PREAMBLE
{\color[rgb]{1,0,0}\put(0,0){\line( 1,1){1}}}% %DIF PREAMBLE
{\color[rgb]{1,0,0}\put(0,1){\line(1,-1){1}}}% %DIF PREAMBLE
\end{picture}% %DIF PREAMBLE
}\hspace*{3pt}}} %DIF PREAMBLE
} %DIF PREAMBLE
\LetLtxMacro{\DIFOaddbegin}{\DIFaddbegin} %DIF PREAMBLE
\LetLtxMacro{\DIFOaddend}{\DIFaddend} %DIF PREAMBLE
\LetLtxMacro{\DIFOdelbegin}{\DIFdelbegin} %DIF PREAMBLE
\LetLtxMacro{\DIFOdelend}{\DIFdelend} %DIF PREAMBLE
\DeclareRobustCommand{\DIFaddbegin}{\DIFOaddbegin \let\includegraphics\DIFaddincludegraphics} %DIF PREAMBLE
\DeclareRobustCommand{\DIFaddend}{\DIFOaddend \let\includegraphics\DIFOincludegraphics} %DIF PREAMBLE
\DeclareRobustCommand{\DIFdelbegin}{\DIFOdelbegin \let\includegraphics\DIFdelincludegraphics} %DIF PREAMBLE
\DeclareRobustCommand{\DIFdelend}{\DIFOaddend \let\includegraphics\DIFOincludegraphics} %DIF PREAMBLE
\LetLtxMacro{\DIFOaddbeginFL}{\DIFaddbeginFL} %DIF PREAMBLE
\LetLtxMacro{\DIFOaddendFL}{\DIFaddendFL} %DIF PREAMBLE
\LetLtxMacro{\DIFOdelbeginFL}{\DIFdelbeginFL} %DIF PREAMBLE
\LetLtxMacro{\DIFOdelendFL}{\DIFdelendFL} %DIF PREAMBLE
\DeclareRobustCommand{\DIFaddbeginFL}{\DIFOaddbeginFL \let\includegraphics\DIFaddincludegraphics} %DIF PREAMBLE
\DeclareRobustCommand{\DIFaddendFL}{\DIFOaddendFL \let\includegraphics\DIFOincludegraphics} %DIF PREAMBLE
\DeclareRobustCommand{\DIFdelbeginFL}{\DIFOdelbeginFL \let\includegraphics\DIFdelincludegraphics} %DIF PREAMBLE
\DeclareRobustCommand{\DIFdelendFL}{\DIFOaddendFL \let\includegraphics\DIFOincludegraphics} %DIF PREAMBLE
%DIF AMSMATHULEM PREAMBLE %DIF PREAMBLE
\makeatletter %DIF PREAMBLE
\let\sout@orig\sout %DIF PREAMBLE
\renewcommand{\sout}[1]{\ifmmode\text{\sout@orig{\ensuremath{#1}}}\else\sout@orig{#1}\fi} %DIF PREAMBLE
\makeatother %DIF PREAMBLE
%DIF COLORLISTINGS PREAMBLE %DIF PREAMBLE
\RequirePackage{listings} %DIF PREAMBLE
\RequirePackage{color} %DIF PREAMBLE
\lstdefinelanguage{DIFcode}{ %DIF PREAMBLE
%DIF DIFCODE_UNDERLINE %DIF PREAMBLE
  moredelim=[il][\color{red}\sout]{\%DIF\ <\ }, %DIF PREAMBLE
  moredelim=[il][\color{blue}\uwave]{\%DIF\ >\ } %DIF PREAMBLE
} %DIF PREAMBLE
\lstdefinestyle{DIFverbatimstyle}{ %DIF PREAMBLE
	language=DIFcode, %DIF PREAMBLE
	basicstyle=\ttfamily, %DIF PREAMBLE
	columns=fullflexible, %DIF PREAMBLE
	keepspaces=true %DIF PREAMBLE
} %DIF PREAMBLE
\lstnewenvironment{DIFverbatim}{\lstset{style=DIFverbatimstyle}}{} %DIF PREAMBLE
\lstnewenvironment{DIFverbatim*}{\lstset{style=DIFverbatimstyle,showspaces=true}}{} %DIF PREAMBLE
\lstset{extendedchars=\true,inputencoding=utf8}

%DIF END PREAMBLE EXTENSION ADDED BY LATEXDIFF

\begin{document}

\title{A Python Decorator for Programmatic and Deterministic Unit Test Code Generation\\}

\author{\IEEEauthorblockN{1\textsuperscript{st} Given Name Surname}
\IEEEauthorblockA{\textit{dept. name of organization (of Aff.)} \\
\textit{name of organization (of Aff.)}\\
City, Country \\
email address or ORCID}
}
\maketitle

\begin{abstract}
Automated software testing is an area of active research, 
    particularly automated generation of unit tests. 
Current and previous work such as generative artificial intelligence, search-based heuristics, and 
    randomization often lead to incomplete coverage and non-deterministic results.
In addition, some of the previous work lacks a natural way for a human expert 
    to provide input to the automated test generator outside the source code itself. 
The author presents a Python decorator to generate deterministic Python 
    unit tests from existing higher level tests via metaprogramming.  
This Python decorator hooks function calls in order to record the 
    metadata (input, output, relevant global variables, and code coverage) 
    of each hooked function. 
The recorded metadata is then used to automatically generate unit 
    tests for each such function.
The author's initial testing confirmed that it can 
    be a force multiplier for the developer, 
    as one passing integration test of the overall 
    system can produce many unit tests.  
This work leverages expert human insight by deriving unit tests
    from as little as one human-designed test.
Further, many test cases may be generated for each test.  
    Python programmers can save significant time and effort using the 
    generated test cases as boilerplate if more test cases are required to 
    increase coverage.  


  \end{abstract} 
\begin{IEEEkeywords}
Python, Automated Test Generation, Metaprogramming
\end{IEEEkeywords}

\section{Introduction}\label{sec:introduction}

\subsection{Software Testing Paradigms}\label{sec:intro-3}

\subsubsection{Test Driven Development}\label{sec:intro-}
Advocates of Test Driven Development (TDD) such a Harry Percival
 \cite{percival2014test} argue that software developers should
follow the TDD process in which they develop software by first writing tests,
then the bare minimum amount of code to make those tests pass,
before beginning the cycle anew with more tests.
On page 22 \cite{percival2014test} Percival advocates both functional and
 unit tests, making the following distinction:
“functional tests test the application from the outside, from the point of view
of the user. Unit tests test the application from the inside,
from the point of view of the programmer.\DIFdelbegin \DIFdel{”
}\DIFdelend \DIFaddbegin \DIFadd{"
}\DIFaddend Later, on page 470 \cite{percival2014test} Percival defines an “integration test\DIFdelbegin \DIFdel{” }\DIFdelend \DIFaddbegin \DIFadd{" }\DIFaddend as
one that depends on and interacts with some external system.
\DIFdelbegin \DIFdel{“Integration test” is sometimes used in common vernacular in the same way that
Percival uses “functional test”.
}\DIFdelend This paper will \DIFdelbegin \DIFdel{do the same, using the terms interchangably.  The author will 
even use }\DIFdelend \DIFaddbegin \DIFadd{use both terms interchangably, in addition to }\DIFaddend the term 
"ad hoc test" in reference to informal tests used by the developer.  
The key asssumption for this paper is that there exists at least one passing 
test above the unit level that calls at least one internal function.  This 
paper will describe how the \DIFaddbegin \DIFadd{author's }\DIFaddend work takes advantage \DIFdelbegin \DIFdel{(to a point) }\DIFdelend of the 
comprehensive nature of higher-level tests in which multiple functions are 
executed, potentially with various inputs.
Previous academic research \cite{causevic2011factors, ramzan2024test} observed
 that industry TDD adoption is lower than desired,
reporting \DIFdelbegin \DIFdel{increased }\DIFdelend longer development cycles, skill issues, and legacy code as 
a few root factors, among others.
\subsubsection{Non-TDD Testing Paradigms}\label{sec:intro-}
In contrast to the TDD methodology in which only the developers write the tests
\cite{axelrod2018unit}, 
others \cite{brown2013agility, shahabuddin2016integration, moe2019comparative} propose a 
more customer-focused testing approaches such as Behavior-Driven Development (BDD)
or Acceptance Test Driven Development (ATDD) focused on \DIFdelbegin \DIFdel{on }\DIFdelend the software's behavior 
from the user's perspective. In these paradigms, non-developers play a role 
in creating the tests, such as a business analyst (\DIFaddbegin \DIFadd{in }\DIFaddend BDD) \cite{barus2019implementation}
or a cross-functional team of developers, business analysts, and testers \cite{pugh2010lean}.
These advocates argue that such feature-first approaches are optimal for accelerating development 
and reducing a product's time to market due to the time-consuming nature of manually
developing unit tests \cite{kahur2023java, shahabuddin2016integration}.
Though Brown \textit{et. al.} acknowledge that integration and unit
testing may go hand-in-hand, they 
ultimately argue in favor of the former before the latter \cite{brown2013agility}.  
Although implied by Brown \textit{et. al.}, Shahabuddin \textit{et. al.} 
\cite{shahabuddin2016integration} go so far as to explicitly argue that unit
testing can be performed \textit{after} initial product delivery to the customer.
This research corresponds with the author’s personal observations that immature 
organizations (or solo developers on small or personal projects)
often don’t adhere to the  TDD approach, and end up writing the tests 
after the development of the code or fail to write tests at all. In such cases,
the hobbyist may focus primarily on function or integration tests
and opt not to develop unit tests, perhaps because the pros of unit tests
do not outweigh the cons compared to a higher level test.
\DIFdelbegin \DIFdel{The main }\DIFdelend \DIFaddbegin \DIFadd{Table \ref{tab1} briefly summarizes a few }\DIFaddend pros and cons \DIFdelbegin \DIFdel{for functionalversus unit testsare summarized very 
briefly Table \ref{tab1}}\DIFdelend \DIFaddbegin \DIFadd{of high level 
(e.g. functional) compared to units tests}\DIFaddend .

\begin{table}[htbp]
    \caption{Pros and Cons of \DIFdelbeginFL \DIFdelFL{Functional }\DIFdelendFL \DIFaddbeginFL \DIFaddFL{High Level }\DIFaddendFL and Unit Tests}
    \begin{center}
    \begin{tabular}{|>{\centering\arraybackslash}m{0.6cm}|>{\centering\arraybackslash}m{3.4cm}|>{\centering\arraybackslash}m{3.4cm}|}
        \hline
        \multirow{2}{*}{} & \textbf{\DIFdelbeginFL \DIFdelFL{Functional }\DIFdelendFL \DIFaddbeginFL \DIFaddFL{High Level }\DIFaddendFL Test(s)} & \textbf{Unit Test(s)} \\
        \hline
        \textbf{Pros} & \begin{itemize}[leftmargin=*]
            \item More relevant for the customer
            \item Efficiently tests modules together
            \item Less dependent on the internal unit implementations
        \end{itemize} & \begin{itemize}[leftmargin=*]
            \item Verifies individual components
            \item Efficient when testing single components
        \end{itemize} \\
        \hline
        \textbf{Cons} & \begin{itemize}[leftmargin=*]
            \item Waste of time when testing small unit changes
            \item May not accurately identify root cause of failure
        \end{itemize} & \begin{itemize}[leftmargin=*]
            \item Time consuming to develop
            \item Easily broken by refactoring code
        \end{itemize} \\
        \hline
    \end{tabular}
    \label{tab1}
    \end{center}
\end{table}

Despite the time-consuming nature of developing unit tests manually, developers of all paradigms
recognize value in \textit{having} unit tests. By definition, any reduction in the cost of unit test creation drives 
up their return on investment.

\subsubsection{Test Coverage}\label{sec:intro-cov}
The term “coverage\DIFdelbegin \DIFdel{” }\DIFdelend \DIFaddbegin \DIFadd{" }\DIFaddend refers to the measure of the completeness of a test.  
Coverage can be measured in various ways, two prominent metrics are branch coverage 
(what percent of if/else branches executed) and line coverage 
(what percent of lines in the function executed during the test) 
\cite{wang2024software}. The \textit{coverage} Python module can
be used to gather this information directly; in fact, it is often paired with 
\textit{pytest} to generate coverage tests for existing tests. \DIFdelbegin \DIFdel{This }\DIFdelend \DIFaddbegin \DIFadd{Without loss of 
generality, this }\DIFaddend paper will focus \DIFdelbegin \DIFdel{on the use of }\DIFdelend \DIFaddbegin \DIFadd{use }\DIFaddend the line coverage metric\DIFdelbegin \DIFdel{, both in implementation and 
as the standard for selecting which execution records to ultimately 
keep for the final generated tests}\DIFdelend .

\subsection{Previous work on automated test creation}\label{sec:intro-3}

Given the cost of creating valuable tests,
the body of academic work on generating them automatically 
via metaprogramming is extensive. Classic algorithms like search 
and randomization feature prominently in earlier work 
\cite{Luk22Pynguin0170}, with generative \DIFdelbegin \DIFdel{AI }\DIFdelend \DIFaddbegin \DIFadd{artificial intelligence (AI)
}\DIFaddend significantly increasing in popularity most recently
\cite{bhatia2023unit,takerngsaksiri2024tdd,wang2024software, kahur2023java}.
These AI methods generate tests based on a variety of inputs, typically the code
itself and some other input such as human prompts 
\cite{lahiri2023interactivecodegenerationtestdriven},
natural language requirements \cite{wang2024software}, or entire projects
\cite{rao2023cat}.  

While the results of such AI-generated tests are promising,
their coverage is not perfect \cite{kahur2023java} and still require significant
developer review and correction \cite{sundqvist2024ai}.  

Lemieux \textit{et. al.} propose a hybrid method, \textit{CodaMosa}, that 
combines search and mutation (classical approaches) with generative AI
\cite{lemieux2023codamosa}.  Their work is impressive but still relies on 
heuristic techniques that the work here does not. This author is also unclear 
what, if any, specialized hardware was required to use \textit{CodaMosa} given
it's reliance on a large language model (LLM).

\subsection{Initial motivation}\label{sec:intro-4}

The author began exploring this concept of automated test case generation
\DIFdelbegin \DIFdel{working on }\DIFdelend \DIFaddbegin \DIFadd{while refactoring }\DIFaddend a personal project, a static analyzer he'd written for C source code.  As the
author did this for fun and personal education, he had not written any unit
tests.  Instead, the author carefully developed a sample input that caused large
portions of the code to execute (e.g. created high code coverage).  This guided
development until the code eventually produced a “good\DIFdelbegin \DIFdel{” }\DIFdelend \DIFaddbegin \DIFadd{" }\DIFaddend output, i.e. \DIFdelbegin \DIFdel{an integration
}\DIFdelend \DIFaddbegin \DIFadd{a functional
}\DIFaddend test.

Unit tests would have greatly eased the refactoring process and the author 
hypothesized that the metaprogramming abilities of Python might empower creation
of unit tests from that sole existing test. This paper describes the author's
effort to prove this hypothesis.

\subsection{Organization of this paper}\label{sec:intro-5}

Section 2 of this paper describes the overall approach to building a
Python decorator to generate unit tests from existing tests. The subsequent
section enumerates both technical and philosophical assumptions made by the
author, as well as alternative approaches when those assumptions may not hold.
Section 4 describes the three different approaches taken by the author to 
evaluate the generated tests, followed by a section dedicated to further 
discussion of this work compared and \DIFdelbegin \DIFdel{constrasted }\DIFdelend \DIFaddbegin \DIFadd{contrasted }\DIFaddend to more related work.  A short
conclusion follows, \DIFdelbegin \DIFdel{trailed by a brief Acknowledgementssection}\DIFdelend \DIFaddbegin \DIFadd{capped off by Acknowledgements}\DIFaddend .

%
 \section{Assumptions}\label{sec:assumptions}

An initial assumption of the author's hypothesis is that all code executing under the
functional test is mostly correct, as the inputs and outputs from each
function called during the functional test would become data for the
unit tests.  

That said, the author acknowledges that software engineers may not 
wish to assume that an apparently working functional test indicates accuracy of 
all internally executed components. Regardless, the creation of such unit 
tests would still provide a syntactically accurate unit
test file for the software engineer to start from, rather than create each unit
test from scratch.  

In addition to saving time from creating the test
boilerplate, the true expense saved is that of manually defining the desired
inputs and correct outputs.  This author believes there is value in this automated
approach even if the generated tests must still be verified for accuracy,
as in the author's experience it was faster in some cases to manually inspect 
the resulting unit test code for accuracy than develop and verify unit tests from scratch.

Another assumption required for successful execution is that the \textit{repr()} method
of each object generates valid Python code that can be used to \DIFdelbegin \DIFdel{re-created }\DIFdelend \DIFaddbegin \DIFadd{re-create }\DIFaddend that
object, or that such a function can be developed and used to temporarily
overwrite a \textit{repr\(\)} method not meeting this requirement.  The author demonstrates
this in the repository code by overwriting the \textit{repr\(\)} method of the
Pandas DataFrame class as shown in Listing~\ref{lst:pandas_repr}.

\DIFdelbegin %DIFDELCMD < \lstinputlisting[language=Python,numbers=left,
%DIFDELCMD < caption={Overwriting incompatible \textit{repr} method},label={lst:pandas_repr},]{examples/pandas\_repr.py}
%DIFDELCMD <  %%%
\DIFdelend \DIFaddbegin \lstinputlisting[language=Python,numbers=left,
caption={Overwriting incompatible \textit{repr} method},label={lst:pandas_repr},]{examples2/pandas\_repr.py}
 \DIFaddend \section{Recording the Execution of a Python Function}\label{sec:approach}

\subsection{Defining the required components to create a unit test}\label{sec:intro-1}

The author set out to record the execution of Python functions
in such a way as to enable exact reproduction of that 
function call in a standalone unit test.  The following components
must be recorded in \DIFdelbegin \DIFdel{some fashion in }\DIFdelend order to do so:
\begin{enumerate}
  \item The function itself \DIFdelbegin %DIFDELCMD < \item %%%
\item%DIFAUXCMD
\DIFdel{The argumentsto the function}\DIFdelend \DIFaddbegin \DIFadd{and it's arguments}\DIFaddend , including \textit{kwargs}
  \item Any relevant global state (e.g. variables)
  \item Exceptions raised
  \item Test coverage
  \item Files/databases read from and/or written to
  \item Data sent/received via a socket
\end{enumerate}

The last two are left for future work, but this paper demonstrates how to capture 
the first five. The subsections that follow discuss exactly how to 
access or determine this info and cache \DIFaddbegin \DIFadd{it }\DIFaddend for subsequent creation of unit tests
with this information.
\subsection{Accessing the function and its arguments}\label{sec:approach-internal-1}

Python enables trivial access to a function and 
its arguments via the concept of decorator functions. \DIFdelbegin %DIFDELCMD < 

%DIFDELCMD < %%%
\DIFdelend Not to be confused with 
the decorator pattern, a Python decorator is a 
function that calls another, thereby permitting the developer to place new code 
before and/or after calling the original “decorated\DIFdelbegin \DIFdel{” }\DIFdelend \DIFaddbegin \DIFadd{" }\DIFaddend function \textit{f}.  
The decorator function has full access to both the decorated 
(or "wrapped" function) \textit{f} as well as all the 
arguments passed to \textit{f}, both \textit{args} and \textit{kwargs}.  In other 
languages this kind of wrapping is often referred to as 
“function hooking\DIFdelbegin \DIFdel{” }\DIFdelend \DIFaddbegin \DIFadd{" }\DIFaddend or "function call interception" 
 \cite{kang2018function}. Any number of decorators can be applied to a function 
in Python, creating a figurative Russian nesting doll of
functions calling functions, each with the ability to access the 
arguments of the function it wraps and modify the 
return value before return. A Python decorator is applied with the \lq@\rq 
symbol as shown on line 20 of Listing~\ref{lst:decorator}\footnote{The author optimized this and other code snippets for display in 
this paper; therefore they may differ slightly from their original 
source found in the associated repository}:

\DIFdelbegin %DIFDELCMD < \lstinputlisting[language=Python,numbers=left,
%DIFDELCMD <   caption={decorator.py: A sample decorator that takes one argument.},label={lst:decorator},]{examples/decorator.py}
%DIFDELCMD < %%%
\DIFdelend \DIFaddbegin \lstinputlisting[language=Python,numbers=left,
  caption={decorator.py: A sample decorator that takes one argument.},label={lst:decorator},]{examples2/decorator.py}
\DIFaddend 

Running the code in Listing~ref{lst:\DIFdelbegin \DIFdel{decorator}\DIFdelend \DIFaddbegin \DIFadd{decorator-output}\DIFaddend } yields the output shown in 
Listing~\ref{lst:decorator-output}. Note the apparently erroneous math (line 1,\DIFdelbegin \DIFdel{7 }\DIFdelend \DIFaddbegin \DIFadd{6 }\DIFaddend of 
Listing~\ref{lst:decorator-output}), as the \textit{add\_ints} function is 
unaware that its first argument was modified by the \text{inner\_most\_decorator}.

\lstset{style=batstyle}

\DIFdelbegin %DIFDELCMD < \lstinputlisting[language=bash,numbers=left,
%DIFDELCMD <   caption={Output of decorator.py},label={lst:decorator-output},]{examples/decorator\_result.txt}
%DIFDELCMD < %%%
\DIFdelend \DIFaddbegin \lstinputlisting[language=bash,numbers=left,
  caption={Output of decorator.py},label={lst:decorator-output},]{examples2/decorator\_result.txt}
\DIFaddend \lstset{style=pythonstyle}

As shown in Listing~\ref{lst:decorator-output}, not only can the Python developer access the function
via the variable \textit{f} (e.g. line 14), the developer also has full
access to the variables passed to \textit{f}, and can make 
arbitrary changes to the arguments in a transparent way (line 11), 
i.e. the calling function would never know the arguments 
were modified before being passed to the callee function.

In addition to access to the function and its arguments,
\DIFdelbegin \DIFdel{decorators provide }\DIFdelend \DIFaddbegin \DIFadd{a decorator provides }\DIFaddend the power to insert code immediately before and after 
the function (lines 8-13, 15, respectively), including leveraging arbitrary
arguments (e.g. \textit{my\_int}) passed to the decorator itself.

A decorator like the one on line 5 of Listing~\ref{lst:decorator} permits
the passing of separate arguments to the decorator itself.  
If \DIFdelbegin \DIFdel{not }\DIFdelend \DIFaddbegin \DIFadd{no }\DIFaddend extraneous argument is needed, that \textit{outermost} decorator
is unnecessary.

The author uses such a decorator (specifically named 
\break
\textit{unit\_test\_generator\_decorator} in the referenced repository) to take
 a “before\DIFdelbegin \DIFdel{” }\DIFdelend \DIFaddbegin \DIFadd{" }\DIFaddend and "after" snapshot of the arguments
before and after the function is called.

\subsection{Accessing relevant global state}\label{sec:approach-internal-2}

In addition to the arguments passed directly to the function, any relevant 
global state must also be captured. "Relevant" here refers only to those 
global values read from and/or written to by the function.
The author's code focuses on variables (e.g. the int \textit{c} in the example above), 
\DIFdelbegin \DIFdel{disregarding imported }\DIFdelend \DIFaddbegin \DIFadd{detecting import }\DIFaddend modules such as \textit{re, os, etc} \DIFdelbegin \DIFdel{, detecting such
modules }\DIFdelend in a separate parsing step 
\DIFdelbegin \DIFdel{.
}\DIFdelend \DIFaddbegin \DIFadd{(see )
}\DIFaddend Access to the global values is non-trivial compared to accessing the function 
and its arguments, but still possible \DIFdelbegin \DIFdel{.  The first step is to use }\DIFdelend \DIFaddbegin \DIFadd{by first using }\DIFaddend the \textit{dis}
module to \DIFdelbegin \DIFdel{programmatically }\DIFdelend disassemble the decorated function \DIFdelbegin \DIFdel{.  This }\DIFdelend \DIFaddbegin \DIFadd{ato determine which global variables 
are ever accessed by the function.  This disassembly }\DIFaddend is only 
required on initial execution of the decoratee as subsequent executions, if any, 
benefit from cached results of the disassembly.
\DIFdelbegin \DIFdel{Programmatic disassembly }\DIFdelend \DIFaddbegin \DIFadd{Disassembly }\DIFaddend of \textit{add\_ints} function during execution is shown
in Listing~\ref{lst:dissas-add-ints}:

\DIFdelbegin %DIFDELCMD < \lstinputlisting[language=TeX,numbers=left,
%DIFDELCMD <   caption={Result of Programmatically Disassembling \textit{add\_ints}},label={lst:dissas-add-ints},]{examples/actual\_disassemble\_add\_ints.txt}
%DIFDELCMD < %%%
\DIFdelend \DIFaddbegin \lstinputlisting[language=TeX,numbers=left,
  caption={Result of Disassembling \textit{add\_ints}},label={lst:dissas-add-ints},]{examples2/actual\_disassemble\_add\_ints.txt}
\DIFaddend 

Note the LOAD\_GLOBAL command on line 6 of Listing 4 to load the value of 
\textit{c}.  Such global names \textit{read from} are saved for later reference.  
Likewise, names and values \textit{written to} via 
the STORE\_GLOBAL commands are also cached for later use.
Of note, the disassembly shown above differs from stand-alone disassembly of 
the same function in the Python interpreter (compare the listing above to the
\textit{\DIFdelbegin \DIFdel{examples}\DIFdelend \DIFaddbegin \DIFadd{examples2}\DIFaddend /disassemble\_decorator\_with\_decorators.txt} file in the 
accompanying repository.)
During actual execution the author's code disassembles only the
\textit{add\_ints} function, after the decorators
have already executed, leaving only the original  \textit{add\_ints}
function for disassembly.
In contrast, disassembling the function in the static, non-executing context of
the Python interpreter reveals the code of all applied decorators.  
Values read from the global state must be recorded 
in order to monkeypatch them in the resulting unit test.
The decorated function must also record modified global values in
order to assert that the state was correctly changed by the function.

\subsection{Detecting exceptions}\label{sec:approach-internal-3}
Detecting exceptions is easiest of the three information capture steps.
Any exception can be detected with the code shown in Listing 
~\ref{exceptions}.

\begin{lstlisting}[language=Python, caption={Catching and recording exceptions}, label={exceptions}]
  try:
    # call the decorated function, e.g.
    f(args, kwargs)
  exception Exception as e:
    # Save the exception type and exception message
  
\end{lstlisting}

\subsection{Determining Test Coverage}\label{sec:approach-internal-4}
The Python \textit{coverage}
module provides support for capturing line (and branch)-based coverage
as shown in Listing~\ref{lst:Coverage} \cite{batchelder2024}.

\DIFdelbegin %DIFDELCMD < \lstinputlisting[language=Python,numbers=left,
%DIFDELCMD <   caption={Calling a function and capturing the test coverage with the \textit{coverage} module.},label={lst:Coverage},]{examples/coverage_example.py}
%DIFDELCMD < %%%
\DIFdelend \DIFaddbegin \lstinputlisting[language=Python,numbers=left,
  caption={Calling a function and capturing the test coverage with the \textit{coverage} module.},label={lst:Coverage},]{examples2/coverage_example.py}
\DIFaddend 

The documented approach to using the \textit{coverage} module 
saves results to a persistent database on the filesystem \DIFaddbegin \DIFadd{as specified with the 
}\textit{\DIFadd{data\_file}} \DIFadd{option}\DIFaddend . However, due to the already high overhead \DIFaddbegin \DIFadd{of }\DIFaddend the 
author's work, he opted \DIFdelbegin \DIFdel{not 
to use the database optionby it separately, opting not to use the default database }\DIFdelend \DIFaddbegin \DIFadd{to specify }\textit{\DIFadd{None}} \DIFadd{for this value to use the in-memory 
option, thereby eliminating such file I/O time }\DIFaddend \cite{batchelder2024}.

\subsection{Summary of the approach}\label{sec:approach-internal-5}

The author uses all the methods discussed above to take a "before" and \DIFdelbegin \DIFdel{“after” 
}\DIFdelend \DIFaddbegin \DIFadd{"after" 
}\DIFaddend snapshot of the arguments and relevant global state of each execution. 

Note that the state of mutable \textit{arguments} must also be captured 
\textit{after} the function executes as called functions may change mutable 
arguments that persist upon return to the caller. The return value or exception 
type and exception message are also captured, in addition to line test coverage. 
For each execution of a given function, an instance of the 
\textit{CoverageInfo} class is created and the fields populated, see 
Listing~\ref{lst:CoverageInfo}.

\DIFdelbegin %DIFDELCMD < \lstinputlisting[language=Python,numbers=left,
%DIFDELCMD <   caption={The CoverageInfo class caches all metadata associated with a single execution.},label={lst:CoverageInfo},]{examples/coverage_info.py}
%DIFDELCMD < %%%
\DIFdelend \DIFaddbegin \lstinputlisting[language=Python,numbers=left,
  caption={The CoverageInfo class caches all metadata associated with a single execution.},label={lst:CoverageInfo},]{examples2/coverage_info.py}
\DIFaddend In addition, one instance of the FunctionMetaData classes shown in 
Listing~\ref{lst:FunctionMetaData} is populated for each decorated function.

\DIFdelbegin %DIFDELCMD < \lstinputlisting[language=Python,numbers=left,
%DIFDELCMD <   caption={The FunctionMetaData class caches all metadata associated with
%DIFDELCMD <    a single function, to include all associated CoverageInfo classes (line 25)},label={lst:FunctionMetaData},]{examples/function\_metadata.py}
%DIFDELCMD < %%%
\DIFdelend \DIFaddbegin \lstinputlisting[language=Python,numbers=left,
  caption={The FunctionMetaData class caches all metadata associated with
   a single function, to include all associated CoverageInfo classes (line 25)},label={lst:FunctionMetaData},]{examples2/function\_metadata.py}
\DIFaddend 

\section{Programmatically Tuning the Decorator}\label{sec:decorator tuning}

\subsection{Selectively bypassing the decorator}\label{sec:tuning-1}

As the reader may imagine, the decorator described above adds significant overhead 
to the execution time of the overall test, as the function call is intercepted,
arguments inspected and copied, etc.  To mitigate this overhead the author provides 
a variety of methods to "bypass" the decorator, i.e. calling the decorated 
function and simply returning the result without further action.

The first and perhaps most obvious approach is to set a coverage threshold such 
that the decorator is effectively disabled once a desired level of coverage 
(e.g. 80\%) is achieved. 

One could also define a specific number 
of executions to capture, after which the decorator would be disabled. 
The author supports either option by setting either the \textit{percent\_coverage} or 
\textit{sample\_count} option to the decorator to the desired value.  Setting 
\textit{percent\_coverage} to a value greater than 100 will capture all 
executions of the decorated function.  Listing~\ref{lst:ProgrammaticallyTuning}
shows this code.
\DIFdelbegin %DIFDELCMD < \lstinputlisting[language=Python,numbers=left,
%DIFDELCMD <   caption={Programmatically "bypassing" the decorator by immediately returning 
%DIFDELCMD <   the function results when desired coverage met},label={lst:ProgrammaticallyTuning},]{examples/programmatictuning.py}
%DIFDELCMD < %%%
\DIFdelend \DIFaddbegin \lstinputlisting[language=Python,numbers=left,
  caption={Programmatically "bypassing" the decorator by immediately returning 
  the function results when desired coverage met},label={lst:ProgrammaticallyTuning},]{examples2/programmatictuning.py}
\DIFaddend 

The careful reader may wonder if the \textit{sample\_count} option may result in 
duplicate tests cases, e.g. five test cases are captured but three are duplicates, 
yielding only three unique test cases.  This would in fact be the case, however, the author 
addressed this by hashing the inputs (global values, args and kwargs) and 
caching the hash in a set \DIFdelbegin \DIFdel{name }\DIFdelend \DIFaddbegin \DIFadd{named }\DIFaddend \textit{hashed\_inputs}.  If a function has
already been called with the exact same inputs, the decorator immediately returns
the result in the same fashion as shown above. Thus, all test cases 
ultimately generated are unique.  This represents the last mitigation
currently in the code.  In future work the author hopes to return the cached 
result/raise the same exception of this execution to save even more overhead.  

It is also important to note that the decorator need only be applied until the 
test cases are generated and the developer is satisfied that the created unit
tests will suffice.  At that point the decorator can be completely removed
from the functional or integration test\footnote{Equivalently, 
\textit{sample\_count} and \textit{percent\_coverage} can both be set to 0.}, 
thereby restoring the runtime of
the original test back to its original, faster runtime.  

\subsection{Selectively keeping specific test cases}\label{sec:tuning-1}
Ideally, \DIFaddbegin \DIFadd{to maximize coverage, }\DIFaddend the same function will be executed multiple 
times during \DIFdelbegin \DIFdel{an ad hoc or function testto maximize coverage}\DIFdelend \DIFaddbegin \DIFadd{a functional test or similarly high level test}\DIFaddend .  That said, 
the tester may wish to preserve 
a minimal number of test cases to avoid too much redundancy.  
One such approach is to drop the current execution
record if the current execution didn’t add any new coverage.  
Conversely, one can drop previous records if the current coverage supersedes
previous coverage.
There are a few ways to minimize duplicate coverage as explored below, 
but the reader should bear in mind:

\begin{enumerate}
  \item that \textit{fully} orthogonal test cases, as measured by coverage, are not feasible 
  (all test cases will cover the initial lines until the first conditional statement) 
  \item test cases with redundant coverage can still hold value
  \item some code such as function calls and regular expressions may have one 
  line in source code, but multiple branches in the underlying library or machine code.
\end{enumerate}

Nevertheless, the value added by redundant tests often plateaus 
\cite{lemieux2023codamosa}, hence the following sections
describe how to reduce such redundancy.  The first section describes
the technique employed by the author to balance memory usage and performance but
with intermediate complexity. The paragraph that follows describes a much simpler and faster
approach but offers poorer performance in that \DIFdelbegin \DIFdel{is }\DIFdelend \DIFaddbegin \DIFadd{it }\DIFaddend rarely selects optimal test 
cases.  For the reader desiring the truly ideal solution, i.e. keep the fewest test cases that achieve
maximum coverage, a note on the proper algorithm to pursue such a goal completes
this section. This method was not selected \DIFdelbegin \DIFdel{to due }\DIFdelend \DIFaddbegin \DIFadd{due to }\DIFaddend its higher memory
requirements and higher complexity of implementation.

\subsubsection{Selected Approach: Individual Subsets}\label{sec:tuning-2}
One can maintain coverage information for previous
executions individually and check the current set of covered lines
against all previous coverage sets.  If the current coverage is
a subset of all existing records, it is discarded.  
Else, all previous records whose coverage is 
a proper subset of the current record are deleted (or \textit{pruned})
and the current record kept. This is more effort than the next approach but
retains fewer records with the same total coverage.  
A simple sample of the individual subset approach is shown in Table 2.

\vskip .2cm
\begin{tabular}{|>{\centering\arraybackslash}m{1.6cm}|>{\centering\arraybackslash}m{1.25cm}|>{\centering\arraybackslash}m{4.5cm}|}
    \hline
    \multirow{2}{*}{\textbf{Execution \#}} & \textbf{Coverage (line \#s)} & \textbf{Action} \\
    \hline
    1 & 1-5 & Keep this record, since it is the first one.\\
    \hline
    2 & 1-7 & Keep this and drop record \#1, as that coverage is only a subset of this coverage.\\
    \hline
    3 & 1-2 & Don’t keep this record; it is a subset of an existing record (\#2)\\
    \hline
\end{tabular}
\captionof{table}{Hypothetical scenario \#1 demonstrating record pruning by checking for subsets}
\vskip .2cm
Unfortunately, this subset approach still leaves room for redundant tests.  
Consider a new scenario:
\vskip .2cm
\begin{tabular}{|>{\centering\arraybackslash}m{1.6cm}|>{\centering\arraybackslash}m{1.25cm}|>{\centering\arraybackslash}m{4.5cm}|}
    \hline
    \multirow{2}{*}{\textbf{Execution \#}} & \textbf{Coverage (line \#s)} & \textbf{Action} \\
    \hline
    1 & 1-5 & Keep this record, since it is the first one.\\
    \hline
    2 & 6-10 & Keep this one, it covers new code.\\
    \hline
    3 & 5-6 & Keep this record, it’s not a subset of any previous record.\\
    \hline
\end{tabular}
\captionof{table}{Hypothetical scenario \#2 demonstrating record pruning by checking for subsets}
\vskip .2cm

Since the subset approach does not aggregate all the coverage into a 
single unified set as in the next approach, the record for execution \#3 would
be kept, but this would be redundant, as executions 1 and 2 already executed
lines 5 and 6, respectively.
The author implemented this approach such that setting the \textit{keep\_subsets}
options to True would not drop redundant records in case the user desired to keep
test cases that produce redundant coverage. The author did maintain a unified
set of coverage as discussed next, but it played no role here.

\subsubsection{Check for new coverage via unified set}\label{sec:tuning-2}
Another approach is to maintain a unified set $U_t$ that is the union of all lines 
covered by all recorded tests from $S_0$ to $S_n$: 

\begin{equation*}
  U_t = S_0 \cup S_1 ...\cup... S_n
\end{equation*}
then only keeping the current record $S_i$ if any currently 
covered lines are not in the unified set:
\begin{equation*}
  S_i \not \subset U_t
\end{equation*}
then updating the unified set with the new coverage information like so:
\begin{equation*}
  U_t \Leftarrow U_t + S_i
\end{equation*} 
With this approach the individual coverage information $S_i$ is not maintained 
($U_t$ is updated, then $S_i$ is discarded), making it impossible to drop superseded 
coverage records over time.  100\% coverage would mean that:
\begin{equation*}
  U_f = U_t
\end{equation*} 
This approach is the simplest of the three described here but suffers from a 
lack of granularity: it's impossible to remove old records if new records have 
better coverage, as specific coverage per test is not recorded.

\subsubsection{Optimal Result by Solving the Minimum Set Cover Problem}\label{sec:tuning-2}
If the developer desires the best solution, e.g. to keep the bare minimum
number of test cases, all records must be cached initially.  
From there, the tester must solve the Minimum Set Cover Problem
\cite{hassin2005better} to select the minimum test cases.
This classic NP-Hard computer science problem involves finding the smallest number
of sets whose union is equal to or greater than (“covers\DIFdelbegin \DIFdel{”}\DIFdelend \DIFaddbegin \DIFadd{"}\DIFaddend ) some universe $U$ of elements.  
In this context, the coverage (set of line number executed) of each test is one set $S_i$,
and the set of all line numbers of a function is the universe $U_f$ to cover.  
As noted, this requires caching the unique coverage of each test, which may 
be prohibitive from a memory standpoint for large applications.
If memory is not a concern, this method may reduce the runtime 
compared to the alternate approaches described above. \footnote{Validating this hypothesis
could be an area for future work.} As the pruning 
only takes place after the tests are complete, the tests are not slowed as much
as the previous methods that prune during the tests.  That said, the sole
pruning operation will take comparatively longer as all the records must be 
analyzed all at once. The reader should be aware that due to the NP-Hard
nature of this problem, typical solutions are heuristic-based.

The author did not implement 
the Minimum Set Cover problem or its weighted variants, but encourages 
the interested reader to do so.  The author records the execution time of each 
execution as the \lq cost \rq. This value could be applied in a weighted 
variant of the Minimum Set Cover problem, i.e. select the "best" test cases
such that coverage is maximized and run time is minimized.

\section{Generating the Unit Tests via Metaprogramming}\label{sec:generating-tests}

After capturing the function metadata as documented in Listings 7 and 8, 
the next step was to programmatically generate the unit tests with all their 
captured test cases and place them in a file named 
\textit{test\_\$function\_name.py}
\DIFdelbegin \DIFdel{. 
}\DIFdelend via a call to \textit{generate\_all\_tests\_and\_metadata} at the end of 
the functional test.
Calling this function eventually executes \textit{auto\_generate\_tests} 
that uses metaprogramming to:
\begin{enumerate}
  \item Build a list of imports
  \item Define global variables constant across all test cases
  \item Convert values to valid (i.e. canonical) Python code
  \item Build a comment specifying the level of coverage achieved\footnote{e.g. line 17 in Listing 13}
  \item Collect parameters to create a parameterized pytest test
  \item Monkeypatch all non-constant relevant globals read from
  \item Assert result is correct
  \item Assert modified globals \DIFaddbegin \DIFadd{and mutable arguments }\DIFaddend are correct
  \item Assert expected exceptions and their messages are correct
  \item Write all of the above to a syntactically correct Python file
  \item Use subprocess to lint and format the result\DIFdelbegin \DIFdel{with }\DIFdelend \DIFaddbegin \DIFadd{, e.g. with }\DIFaddend \textit{black \DIFdelbegin %DIFDELCMD < \MBLOCKRIGHTBRACE %%%
\DIFdelend and\DIFdelbegin \textit{\DIFdel{ruff}}
%DIFAUXCMD
\DIFdelend \DIFaddbegin \DIFadd{/or ruff}}
\DIFaddend \end{enumerate}

To aid the developer's troubleshooting efforts, the contents of each FunctionMetaData
class are also dumped to their own .json file, one per decorated function (hence
the \textit{and\_metadata} suffix noted above) \DIFaddbegin \footnote{\DIFadd{This includes 
"untestable" test cases, i.e. those that could not be represented as canonical strings
but may be valuable to the developer regardless}}\DIFaddend .  This file permits the 
developer to easily inspect the inputs, outputs, and other data
associated with the recording without the distraction of the test code.

Thus, if the function \textit{foo} is decorated with \textit{unit\_test\_decorator},
the generated unit test code can be found in \textit{test\_foo.py}, and the 
metadata can be found in a JSON file name named \textit{foo.json}

In addition to all of the above, significant development efforts went into 
serializing and deserializing Python code to and from strings.  Though
certainly not a panacea, the work here advances that done by 
Lemueux \textit{et. al.} \cite{lemieux2023codamosa} not only in terms of 
automated testing but also in terms of serialization capabilities.




 \section{Tester, Test Thyself}\label{sec:evaluation}

The decorator was evaluated informally with three separate approaches.
The author began by crafting simple but demonstrative test cases 
(found in the \textit{tests/} folder in the associated repository.) Next,
the author applied the concept of programmatically generated unit tests
to the code itself with positive results. With two key logical checks it
was possible to apply the decorator to its very own support code.  Those
checks were to:

\begin{enumerate}
  \item Immediately return if the pytest module was loaded, indicating a unit (not integration) test is currently being executed
  \item Immediately return if the call stack contained a loop, e.g. A$\rightarrow$B$\rightarrow$A
\end{enumerate}

These checks are shown in Listing~\ref{lst:Limit Recursion}.

\DIFdelbegin %DIFDELCMD < \lstinputlisting[language=Python,numbers=left,
%DIFDELCMD <   caption={limit\_recursion.py},label={lst:Limit Recursion},]{examples/limit_recursion.py}
%DIFDELCMD < %%%
\DIFdelend \DIFaddbegin \lstinputlisting[language=Python,numbers=left,
  caption={limit\_recursion.py},label={lst:Limit Recursion},]{examples2/limit_recursion.py}
\DIFaddend 

\DIFdelbegin \DIFdel{the carefule }\DIFdelend \DIFaddbegin \DIFadd{Careful }\DIFaddend examination of the call stack was required to limit 
recursive decorators to just one level. Finally, as previously mentioned, 
the author applied his work to another project involving parsing C code. 
These three approaches are described detail below:

\subsection{Manually Created Tests}\label{sec:eval-1}
The author created a variety of tests to ensure the unit test generation 
code functioned properly. The Procedural Division example is explained in depth
below and the others are briefly summarized.

The \textit{tests/example\_procedural\_division} \DIFdelbegin \DIFdel{test was intended 
to test }\DIFdelend \DIFaddbegin \DIFadd{tests }\DIFaddend procedural (as opposed 
to object-oriented or functional) code that:  

\begin{enumerate}
    \item Returned a string given two ints
    \item Wrote to a global variable
    \item Raised two different types of exceptions
\end{enumerate}

The author wrote a \textit{divide\_ints} function as shown 
in Listing~\ref{lst:Divide Ints}.

\DIFdelbegin %DIFDELCMD < \lstinputlisting[language=Python,numbers=left,
%DIFDELCMD <   caption={divide\_ints.py},label={lst:Divide Ints},]{examples/divide\_ints\_1.py}
%DIFDELCMD < %%%
\DIFdelend \DIFaddbegin \lstinputlisting[language=Python,numbers=left,
  caption={divide\_ints.py},label={lst:Divide Ints},]{examples2/divide\_ints\_1.py}
\DIFaddend 

\DIFdelbegin \DIFdel{Then }\DIFdelend \DIFaddbegin \DIFadd{The author then }\DIFaddend wrote an ad-hoc "test" that calls 
\textit{divide\_ints} with a variety of inputs, but no assertions
as shown in List~\ref{lst:Test Divide Ints}.  Note the \DIFdelbegin \DIFdel{useof }\DIFdelend \DIFaddbegin \DIFadd{use of }\DIFaddend erroneous inputs
to cover the error handling code.

\DIFdelbegin %DIFDELCMD < \lstinputlisting[language=Python,numbers=left,
%DIFDELCMD <   caption={test divide\_ints()},label={lst:Test Divide Ints},]{examples/divide\_ints\_2.py}
%DIFDELCMD < %%%
\DIFdelend \DIFaddbegin \lstinputlisting[language=Python,numbers=left,
  caption={test divide\_ints()},label={lst:Test Divide Ints},]{examples2/divide\_ints\_2.py}
\DIFaddend 

Running this code with pytest as displayed in Listing 6

\begin{lstlisting}[
  language=bash, 
  caption={Executing example to create unit tezst},
  label={lst:rundivide}]
    $ python divide_ints.py
\end{lstlisting}

produces the following \textit{test\_divide\_ints.py} containing the following 
unit test (modified only slightly below to reduce line breaks):

\DIFdelbegin %DIFDELCMD < \lstinputlisting[language=Python,numbers=left,
%DIFDELCMD <   caption={test divide\_ints},label={lst:Test Divide Ints},]{examples/divide\_ints\_3.py}
%DIFDELCMD < %%%
\DIFdelend \DIFaddbegin \lstinputlisting[language=Python,numbers=left,
  caption={test divide\_ints},label={lst:Test Divide Ints},]{examples2/divide\_ints\_3.py}
\DIFaddend 

Lines 18-20 set up the parameterization decorator, defining the inputs to each test,
and the inputs to each are defined in a list of tuples spanning lines 23-33, 
34-44, 45-57.  The actual test function follows starting on line 60, monkeypatches
the required ERROR\_CODE global variable on lines 75-76, tests for expected 
exceptions on lines 77-80, and only calls the function on line 82 if it expects 
no exception. The result is verified on lines 82-83 \footnote{\DIFdelbegin \DIFdel{due }\DIFdelend \DIFaddbegin \DIFadd{Due 
}\DIFaddend to the complexity of converting from Python object to string
(and sometimes back again), the author was forced due to time \DIFdelbegin \DIFdel{contraints }\DIFdelend \DIFaddbegin \DIFadd{constraints }\DIFaddend to
rely on the \textit{eval} function to convert strings to valid Python.  
The author acknowleges this is a bad practice and hopes to fix it in
future revisions.}. An unexpected exception (or incorrect 
(exception, exception message) 
pair) would cause the unit test to fail.  Finally, global variables expected to be modified 
are checked for correct values on lines 85-92.  Note that not all test cases 
were retained due to the deduplication logic discussed in Section~\ref{sec:tuning-1}.

Executing the unit test is simple as shown in Listing~\DIFdelbegin \DIFdel{\ref{Running one of the generated unit tests}}\DIFdelend \DIFaddbegin \DIFadd{\ref{Running}}\DIFaddend \footnote{Due to the verbose way pytest prints the 
all the parameters of  parameterized tests such as these, the author removed
them for the sake of simpler display.  The reader is encouraged to run the code 
as shown on their own machine}.

\DIFdelbegin %DIFDELCMD < \begin{lstlisting}[language=bash, numbers=left, caption={Running one of the generated unit tests}]
%DIFDELCMD < %%%
\DIFdelend \DIFaddbegin \begin{lstlisting}[language=bash, numbers=left, caption={Running}]
pytest -s -v test_divide_ints.py
========== test session starts ==========
platform win32 -- Python 3.11.7, pytest-7.4.4, pluggy-1.4.0 -- <@\textcolor{orange}{PATH REDACTED}@>\.venv\Scripts\python.exe
cachedir: .pytest_cache
rootdir: <@\textcolor{orange}{PATH REDACTED}@>\tests\example_procedural_division
plugins: cov-5.0.0
collected 4 items

test_divide_ints.py::test_divide_ints[<@\textcolor{orange}{Test \#1 arguments SNIPPED}@>] <@\textcolor{green}{PASSED}@>
test_divide_ints.py::test_divide_ints[<@\textcolor{orange}{Test \#2 arguments SNIPPED}@>] <@\textcolor{green}{PASSED}@>
test_divide_ints.py::test_divide_ints[<@\textcolor{orange}{Test \#3 arguments SNIPPED}@>] <@\textcolor{green}{PASSED}@>
test_divide_ints.py::test_divide_ints[<@\textcolor{orange}{Test \#4 arguments SNIPPED}@>] <@\textcolor{green}{PASSED}@>
\end{lstlisting}

The reader is encouraged to study the other examples in the repository.
All examples can be executed via the 
\textit{tests/test\_\DIFdelbegin \DIFdel{all.*}\DIFdelend \DIFaddbegin \DIFadd{all*}\DIFaddend } scripts.  The reader can inspect those scripts
to determine how to run each test individually. 
They are summarized very briefly below:

The \textit{tests/example\_fizzbuzz} \DIFdelbegin \DIFdel{test is nearly identical to the 
test described above: tests }\DIFdelend \DIFaddbegin \DIFadd{tests I/O to a simple function and }\DIFaddend modification of a 
global value.

The \textit{tests/example\_all\_types} \DIFdelbegin \DIFdel{test was designed to test
}\DIFdelend \DIFaddbegin \DIFadd{tests
}\DIFaddend the code against a variety of built-in Python types, 
such as ints, strings, sets, lists, tuples, and dictionaries.

The \textit{tests/example\_pass\_by\_assignment} \DIFdelbegin \DIFdel{test was designed to test
}\DIFdelend \DIFaddbegin \DIFadd{tests
}\DIFaddend functions that modify mutable arguments such as lists. Not a comprehensive test,
but demonstrates that the concept works on one such example.

The \textit{tests/example\_oo\_car} \DIFdelbegin \DIFdel{test was designed to test
}\DIFdelend \DIFaddbegin \DIFadd{tests }\DIFaddend object-oriented code. 
\DIFdelbegin \DIFdel{Not complete at the time of submission, but worked in a 
previous branch prior to adding support for the test above.
The author hopes to restore this feature to full functionality in future work.
}\DIFdelend \subsection{Application of this code to itself}\label{sec:eval-2}

The author naturally sought to determine the effectiveness of automatic
test code generation against itself.  \DIFdelbegin \DIFdel{Although not 
every function in the referenced code was decorated for automatic testing, 
as of this writing five functions in }\DIFdelend \DIFaddbegin \DIFadd{This attempt at self-testing code created 
an intractable problem for this author due to the recursion in the call stack
combined with lack of proper }\DIFaddend \textit{\DIFdelbegin \DIFdel{unit\_test\_generator\_decorator.py}\DIFdelend \DIFaddbegin \DIFadd{repr}\DIFaddend } \DIFdelbegin \DIFdel{were successfully decorated
}\footnote{\DIFdel{with the exception of a few OOP relate tests, see above.}}%DIFAUXCMD
\addtocounter{footnote}{-1}%DIFAUXCMD
\DIFdelend \DIFaddbegin \DIFadd{functions.  That said, the author did 
successfully decorate three of the support functions for "self-testing"}\DIFaddend ,
resulting in automatically generated, successfully passing unit tests for this
\DIFaddbegin \DIFadd{portions of this }\DIFaddend very code.  Using the previous Procedural Division example (and others) 
the reader should see the following test files created in addition to 
\textit{test\_divide\_ints.py}

\begin{enumerate}
    \item \textit{test\_coverage\_str\_helper.py}
    \item \DIFdelbegin \textit{\DIFdel{test\_meta\_program\_function\_call.py}}
    %DIFAUXCMD
%DIFDELCMD < \item %%%
\item%DIFAUXCMD
\DIFdelend \textit{test\_normalize\_args.py}
    \item \textit{test\_update\_global.py}
\DIFdelbegin %DIFDELCMD < \item %%%
\item%DIFAUXCMD
\textit{\DIFdel{test\_update\_metadata.py}}
%DIFAUXCMD
\DIFdelend \end{enumerate}

These tests \DIFdelbegin \DIFdel{also }\DIFdelend pass when run with this command in 
\linebreak
\textit{tests/example\_procedural\_division/} directory:
\begin{lstlisting}[language=bash, caption={Running all generated unit
   tests for the division example}]
    pytest -s -v .
\end{lstlisting}

\subsection{Application of this code to an external project}\label{sec:eval-2}
To validate this work on an external codebase, the author applied 
this project to a single function (due to time constraints) of his C code
parser that inspired it.  The author applied this
decorator to a Python function designed to extract the name of any C functions 
from a provided line of code, then ran the sole function test that called 
the decorated function extensively with various and redundant inputs.
The applied decorator worked as desired, including deduplicating results as
previously discussed, rapidly capturing over three dozen unique test cases.
Upon visually inspecting the inputs and outputs, the author 
nearly instantly identified two bugs in the decorated function.
After patching both bugs revealed by the unit tests, he re-ran the function test, 
and inspected the newly created unit tests. At that point he was confident that
the generated unit test and dozens of tests cases sufficed for unit testing apart 
from the functional test.

\DIFdelbegin \DIFdel{At that point}\DIFdelend \DIFaddbegin \DIFadd{Satisfied}\DIFaddend , he removed the decorator, restoring the functional test to its 
original runtime. Thus, under ideal situations, the tester only pays the 
decoration penalty as few as two times per function.  Of course, multiple 
functions can be decorated at once.
 \section{Discussion}\label{sec:discussion}


In this paper and accompanying code the author
has proved his hypothesis that the metaprogramming abilities of 
Python can be used to programmatically create multiple 
unit tests from even a single existing test.

The decorator and supporting code discussed above
(and provided in the accompanying repository)
\DIFdelbegin \DIFdel{was }\DIFdelend \DIFaddbegin \DIFadd{were }\DIFaddend written in Python 3.11 and tested on a 
Windows 10 laptop with 16GB of memory and 6 core i7-9750H @2.60Ghz CPU.
The author did not test it on *nix environments but given the
cross-platform support of Python, the author believes it should run in *nix 
environments with no more than minor modifications. The author releases it
under the GNU LGPL open-source license.

\subsection{Areas for future work}\label{sec:tuning-2}
As partly noted above, future work should: 
\begin{enumerate}
  \item Reduce reliance on \textit{eval}
  \item Add support for file and socket I/O
  \item \DIFdelbegin \DIFdel{Complete OOP support
  }%DIFDELCMD < \item %%%
\item%DIFAUXCMD
\DIFdelend Compare runtime performance of various algorithms for keeping minimal subsets of tests
  \item Add multi-thread and multi-process testing and support
  \item Investigate applicability of these ideas to other OOP languages such as C\#, Go, and Zig
  \item Verify cross-platform compatibility
  \item Determine backwards compatibility with older Python versions
\end{enumerate}

\subsection{Limitations}\label{sec:limitations}
As noted in Section~\ref{sec:assumptions}, in order to 
generate unit tests as described in this paper, the 
process must be boot-strapped with a initial passing test.

Additionally, for very large applications, or large
functions with complex inputs, memory constraints may 
limit the number of functions that can be decorated at once,
which would slow down the process of creating 
a large number of unit tests.

\subsection{Related Work}\label{sec:related-work}
This work is perhaps most similar to 
\cite{lahiri2023interactivecodegenerationtestdriven}, though
differs in that rather than prompt users interactively for sample inputs and
outputs, it automatically detects such inputs, outputs, and changes in global
state by monitoring actual execution of the code.  It is also worth noting 
that the work presented here could easily be applied to AI or otherwise 
automatically generated tests.  Although in one paper the AI-generated end-to-end tests 
were written in Java \cite{leotta2024ai}, analogously created Python tests could
be used with the work presented here to generate previously non-existent 
tests at the unit level.

In the current research this author did not discover a deterministic 
unit test generation paradigm that monitored other types of
tests from which to create unit tests.
 \DIFaddbegin \section{\DIFadd{Conclusion}}\label{sec:conclusion}
\DIFaddend In this paper the author presented
a Python decorator \DIFdelbegin \DIFdel{leveraged }\DIFdelend \DIFaddbegin \DIFadd{leveraging }\DIFaddend previously designed test(s) to directly create 
unit tests at minimal development cost.
The trade-off is that this approach increases execution time of the function 
tests and\DIFaddbegin \DIFadd{, }\DIFaddend like AI-generated tests, such tests would still require review by the 
developer. The latter is true of any automatically generated test and 
the author proposed and demonstrated a variety of approaches to 
mitigate the increased runtime for existing tests.
\DIFdelbegin \section{\DIFdel{Conclusion}}%DIFAUXCMD
\addtocounter{section}{-1}%DIFAUXCMD
%DIFDELCMD < \label{sec:conclusion}
%DIFDELCMD < %%%
\DIFdelend 

The author proved his initial hypothesis that 
that unit tests could be
created programmatically by monitoring and capturing 
metadata from existing tests.

Although much work remains, the author maintains an optimistic outlook
that the testing paradigm presented here could prove valuable in the 
world of automated test generation due to the variety of successful 
tests noted above.  By generating unit tests from functional tests, 
not only are the unit tests created essentially for free,
but the quality (i.e. coverage) of other tests generated via BDD or ATDD 
processes can be assessed. This could enable rapid comparison
of the relative test coverage of various testing paradigms.

The code is available in a single Python file to permit 
easy integration into any supported Python project.
The author wishes to give back to the computing community
and therefore offers this project free and open source
in the hopes that it will be found useful and lay the ground
work for future \DIFdelbegin \DIFdel{related }\DIFdelend research and application.
The author hopes the concepts described herein will be 
applied in other languages with strong metaprogramming support
such as C\#, Go, Zig, etc. Further information on the code and  documentation, please
find it online at
\begin{center}
  \textbf{LINK REDACTED FOR DOUBLE BLIND REVIEW}
\end{center}


 \section{Acknowledgements}\label{sec:acknowledgements}

Redacted for double-blind review; will populate if accepted.  e.g.
The author wishes to thank Drs. \_\_\_\_\_ \_\_\_\_\_\_ and \_\_\_\_\_\_\_ \_\_\_\_\_\_\_\_
for their assistance in technical writing, Mr. \_\_\_\_\_ \_\_\_\_\_\_
for his input on the abstract, and Mr. \_\_\_\_\_ \_\_\_\_\_\_ for sharing his pytest
expertise.
\DIFaddbegin 



 \newpage
\DIFaddend 

\bibliographystyle{ieeetran}
\bibliography{related}

\end{document}
