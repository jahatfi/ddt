% Add option `kerning` to `microtype`. Since `acmart` already loads `microtype`
% by itself, we need to add this option here to hook into the package-loading
% process.
\PassOptionsToPackage{kerning}{microtype}
%
\documentclass[%
  anonymous=true,%
  authordraft=false,%
  format=sigconf,%
  review=false,%
  screen=true,%
  timestamp=false,%
  pbalance,%
]{acmart}

\copyrightyear{2024}


\usepackage{defs}
\usepackage{listings}
\lstset{escapeinside={<@}{@>}}
\usepackage{amsmath}
\usepackage{xcolor}
%\usepackage{lipsum,multicol}
\graphicspath{{./}{./img/}}

\newcommand\numModules{\num{118}\xspace}
\newcommand\numIterations{\num[round-precision=2]{30}\xspace}
\newcommand\timeout{\SI{600}{\second}\xspace}
\newcommand\numProjects{\num[round-precision=2]{17}\xspace}
\newcommand\avgCoverage{\SI{67.96650716076552}{\percent}\xspace}
\newcommand\avgCoverageDynaMOSA{\SI{67.96650716076552}{\percent}\xspace}
\newcommand\avgCoverageMIO{\SI{66.99329239952593}{\percent}\xspace}
\newcommand\avgCoverageMOSA{\SI{67.78516671267488}{\percent}\xspace}
\newcommand\avgCoverageRandom{\SI{63.56828612377363}{\percent}\xspace}
\newcommand\avgCoverageWS{\SI{66.9088489056397}{\percent}\xspace}
\newcommand\avgCoverageWSA{\SI{67.53200462581992}{\percent}\xspace}

\setlength {\marginparwidth }{2cm}
\begin{document}
%\begin{multicols}{2}

\title{A Python Decorator for Programmatic and Deterministic Unit Test Code Generation}

\author{James Hatfield}
\email{jahatfi@gmail.com}
\orcid{0009-0003-1677-0026}
\affiliation{
  \state{Maryland}
  \country{United States of America}
}


  
  \ccsdesc[300]{Software and its engineering~Software maintenance tools}
  \ccsdesc[500]{Software and its engineering~Software testing and debugging}
  \ccsdesc[500]{Software and its engineering~Empirical software validation}
  \ccsdesc[100]{Software and its engineering~Object oriented frameworks}
  \ccsdesc[100]{Software and its engineering~Object oriented development}

\begin{abstract}
  %
  % Context
  Automated software testing is an area of active research, 
  particularly automated generation of unit tests. 
  %
  % Problem
  Current and previous work such as generative AI, search-based heuristics, and 
  randomization often led to incomplete coverage and non-deterministic results.
  %
  In addition, some of the previous work lacks a natural way for a human expert 
  to provide input to the automated test generator outside the source code itself. 
  % Insight
  The author presents a Python decorator to generate deterministic Python 
  unit tests from existing higher level tests via metaprogramming.  
  %
  This Python decorator hooks function calls in order to record the 
  metadata (input, output, relevant global variables, and code coverage) 
  of each hooked function. 
  %
  The recorded metadata is then used to automatically generate unit 
  tests for each such function.
  %
  % Contribution
  The author's initial testing confirmed that it can 
  be a force multiplier for the developer, 
  as one passing integration or similar test of the overall 
  system can produce many unit tests.  
  %
  As the initial test is designed by the developer, it provides a natural 
  approach to provide expert human insight that is leveraged to create 
  higher quality unit tests.
  %
  Further, many test cases may be generated for each test.  
  Python programmers can save significant time and effort using the 
  generated test cases as boilerplate if more test cases are required to 
  increase coverage.  
  %

\end{abstract}

\keywords{Python, Automated Test Generation, Metaprogramming}

\maketitle

\section{Introduction}\label{sec:introduction}

\subsection{Software Testing Paradigms}\label{sec:intro-3}

\subsubsection{Test Driven Development}\label{sec:intro-}
Advocates of Test Driven Development (TDD) such a Harry Percival
 \cite{percival2014test} argue that software developers should
follow the TDD process in which they develop software by first writing tests,
then the bare minimum amount of code to make those tests pass,
before beginning the cycle anew with more tests.
%
On page 22 \cite{percival2014test} Percival advocates both functional and
 unit tests, making the following distinction:
“functional tests test the application from the outside, from the point of view
of the user. Unit tests test the application from the inside,
from the point of view of the programmer.”
%
Later, on page 470 \cite{percival2014test} Percival defines an “integration test” as
one that depends on and interacts with some external system.
%
“Integration test” is sometimes used in common vernacular in the same way that
Percival uses “functional test”.
%
This paper will do the same, using the terms interchangably.  The author will 
even use the term "ad hoc test" in reference to informal tests used by the developer.  
The key asssumption for this paper is that there exists at least one passing 
test above the unit level that calls at least one internal function.  This 
paper will describe how the work takes advantage (to a point) of the 
comprehensive nature of higher-level tests in which multiple functions are 
executed, potentially with various inputs.
%
Previous academic research \cite{causevic2011factors, ramzan2024test} observed
 that industry TDD adoption is lower than desired,
reporting increased longer development cycles, skill issues, and legacy code as 
a few root factors, among others.
%
\subsubsection{Non-TDD Testing Paradigms}\label{sec:intro-}
In contrast to the TDD methodology in which only the developers write the tests
\cite{axelrod2018unit}, 
others \cite{brown2013agility, shahabuddin2016integration, moe2019comparative} propose a 
more customer-focused testing approaches such as Behavior-Driven Development (BDD)
or Acceptance Test Driven Development (ATDD) focused on on the software's behavior 
from the user's perspective. In these paradigms, non-developers play a role 
in creating the tests, such as a business analyst (BDD) \cite{barus2019implementation}
or a cross-functional team of developers, business analysts, and testers \cite{pugh2010lean}.
These advocates argue that such feature-first approaches are optimal for accelerating development 
and reducing a product's time to market due to the time-consuming nature of manually
developing unit tests \cite{kahur2023java, shahabuddin2016integration}.
Though Brown \textit{et. al.} acknowledge that integration and unit
testing may go hand-in-hand, they 
ultimately argue in favor of the former before the latter \cite{brown2013agility}.  
Although implied by Brown \textit{et. al.}, Shahabuddin \textit{et. al.} 
\cite{shahabuddin2016integration} go so far as to explicitly argue that unit
testing can be performed \textit{after} initial product delivery to the customer.
%
This research corresponds with the author’s personal observations that immature 
organizations (or solo developers on small or personal projects)
often don’t adhere to the  TDD approach, and end up writing the tests 
after the development of the code or fail to write tests at all. In such cases,
the hobbyist may focus primarily on function or integration tests
and opt not to develop unit tests, perhaps because the pros of unit tests
do not outweigh the cons compared to a higher level test.
%
The main pros and cons for functional versus unit tests are summarized very
briefly Table 1.
\vskip .2cm
\noindent\begin{tabular}{|>{\centering\arraybackslash}m{0.6cm}|>{\centering\arraybackslash}m{3.4cm}|>{\centering\arraybackslash}m{3.4cm}|}
    \hline
    \multirow{2}{*}{} & \textbf{Functional Test(s)} & \textbf{Unit Test(s)} \\
    \hline
    \textbf{Pros} & \begin{itemize}[leftmargin=*]
        \item More relevant for the customer
        \item Efficiently tests modules together
        \item Less dependent on the internal unit implementations
    \end{itemize} & \begin{itemize}[leftmargin=*]
        \item Verifies individual components
        \item Efficient when testing single components
    \end{itemize} \\
    \hline
    \textbf{Cons} & \begin{itemize}[leftmargin=*]
        \item Waste of time when testing small unit changes
        \item May not accurately identify root cause of failure
    \end{itemize} & \begin{itemize}[leftmargin=*]
        \item Time consuming to develop
        \item Easily broken by refactoring code
    \end{itemize} \\
    \hline
\end{tabular}
\captionof{table}{Pros and Cons of Functional and Unit Tests}
Despite the time-consuming nature of developing unit tests manually, both
TDD and integration-test-first proponents recognize value in \textit{having}
unit tests. By definition, any reduction in the cost of unit test creation drives 
up their return on investment.

\subsubsection{Test Coverage}\label{sec:intro-cov}
The term “coverage” refers to the measure of the completeness of a test.  
Coverage can be measured in various ways, two prominent metrics are branch coverage 
(what percent of if/else branches executed) and line coverage 
(what percent of lines in the function executed during the test) 
\cite{wang2024software}. The \textit{coverage} Python module can
be used to gather this information directly; in fact, it is often paired with 
\textit{pytest} to generate coverage tests for existing tests. This paper will focus 
on the use of the line coverage metric, both in implementation and 
as the standard for selecting which execution records to ultimately 
keep for the final generated tests.

\subsection{Previous work on automated test creation}\label{sec:intro-3}

Given the cost of creating valuable tests,
the body of academic work on generating them automatically 
via metaprogramming is extensive. Classic algorithms like search 
and randomization feature prominently in earlier work 
\cite{Luk22Pynguin0170}, with generative AI significantly increasing in 
popularity most recently
\cite{bhatia2023unit,takerngsaksiri2024tdd,wang2024software, kahur2023java}.
%
These AI methods generate tests based on a variety of inputs, typically the code
itself and some other input such as human prompts 
\cite{lahiri2023interactivecodegenerationtestdriven},
natural language requirements \cite{wang2024software}, or entire projects
\cite{rao2023cat}.  

While the results of such AI-generated tests are promising,
their coverage is not perfect \cite{kahur2023java} and still require significant
developer review and correction \cite{sundqvist2024ai}.  

Lemieux \textit{et. al.} propose a hybrid method, \textit{CodaMosa}, that 
combines search and mutation (classical approaches) with generative AI
\cite{lemieux2023codamosa}.  Their work is impressive but still relies on 
heuristic techniques that the work here does not. This author is also unclear 
what, if any, specialized hardware was required to use \textit{CodaMosa} given
it's reliance on a large language model (LLM).

\subsection{Initial motivation}\label{sec:intro-4}

The author began exploring this concept of automated test case generation
working on a personal project, a static analyzer he'd written for C source code.  As the
author did this for fun and personal education, he had not written any unit
tests.  Instead, the author carefully developed a sample input that caused large
portions of the code to execute (e.g. created high code coverage).  This guided
development until the code eventually produced a “good” output, i.e. an integration
test.
  
Unit tests would have greatly eased the refactoring process and the author 
hypothesized that the metaprogramming abilities of Python might empower creation
of unit tests from that sole existing test. This paper describes the author's
effort to prove this hypothesis.

\subsection{Organization of this paper}\label{sec:intro-5}

Section 2 of this paper describes the overall approach to building a
Python decorator to generate unit tests from existing tests. The subsequent
section enumerates both technical and philosophical assumptions made by the
author, as well as alternative approaches when those assumptions may not hold.
Section 4 describes the three different approaches taken by the author to 
evaluate the generated tests, followed by a section dedicated to further 
discussion of this work compared and constrasted to more related work.  A short
conclusion follows, trailed by a brief Acknowledgements section.

% vim: spelling=en_US
\section{Assumptions}\label{sec:introduction}

the author determined that unit tests could be
created programmatically by assuming that all code executing under the
functional test was in fact correct, then using the inputs and outputs from each
function called during the functional test as the inputs and outputs for the
unit tests.  

The author acknowleges that software engineers may not 
wish to assume that a apparently working functional test indicates accuracy of 
all internally executed components. Regardless, the creation of such unit 
tests would still provide a syntactically accurate unit
test file for the software engineer to start from, rather than create each unit
test from scratch.  

In addition to saving time from creating the test
boilerplate, the true expense saved is that of manually defining the desired
inputs and correct outputs.  This author believes there is value in this
approach even if the generated tests must still be verified for accuracy, as it
could be faster to manually inspect code for accuracy than develop and verify
such code from scratch.

Another assumption required for successful execution is that the \textit{repr()} method
of each object generates valid Python code than can be used to re-created that
object, or that such a function can be developed and used to temporarily
overwrite a \textit{repr\(\)} method not meeting this requirement.  The author demonstrates
this in the repository code by overwriting the \textit{repr\(\)} method of the
Pandas DataFrame class in Listing 1.

 
 \lstinputlisting[%
 language=Python,%
 numbers=left,
 caption={Overwriting incompatible \textit{repr} method},%
 label={lst:pandas_repr},%
]{examples/pandas\_repr.py}

\section{Recording the Execution of a Python Function}\label{sec:approach}
% Overview of concepts
The author set out to record the execution of Python functions
in such a way as to enable exact reproduction of that 
function call at at later time.  The following components
must be recorded in some fashion in order to do so:
\begin{enumerate}
  \item The function itself
  \item The arguments to the function, including kwargs
  \item Any relevant global state \(e.g. variables\)
  \item Exceptions raised
  \item Test coverage
  \item Files/databases read from and/or written to
  \item Data sent/received via a socket
\end{enumerate}

The last two are left for future work, but this paper demonstrates how to use 
the initial components. The subsections that follow discuss exactly how to 
access or determine this info and cache for subsequent creation of unit tests
with this information.
%%
%% General information about The decorator
%%
\subsection{Accessing the function and its arguments}\label{sec:approach-internal-1}

Python enables trivial access to a function and 
its arguments by another function via the concept of 
decorator functions.

Not to be confused with the decorator pattern, a Python decorator is simply a 
function that calls another, thereby permitting the developer to place new code 
before and/or after calling the original “decorated” function.  
The decorator function has full access to both the decorated 
(or "wrapped" function), \textit{f} as well as all the 
arguments passed to \textit{f}, both args and kwargs.  In other 
languages this kind of wrapping is often referred to as 
“function hooking” or "function call interception" 
 \cite{kang2018function}. Any number of decorators can be applied to a function 
in Python, creating a figurative Russian nesting doll of
functions calling functions, each will the ability to access the 
arguments and functions of the function below it, and modify the 
return value before return.

% Show the decorator syntax and what it actually looks like under the hood,
% noting that f, args, and kwargs are accessible
A Python decorator is applied with the '@' symbol as shown below:

\lstinputlisting[%
  language=Python,%
  numbers=left,
  caption={decorator.py: A sample decorator that takes one argument.},%
  label={lst:decorator},%
]{examples/decorator.py}
\newpage
Running the code above yields:

\lstinputlisting[%
  language=bash,%
  caption={Output of decorator.py},%
  label={lst:decorator},%
]{examples/decorator\_result.txt}

As demonstrated above, not only can the Python developer access the function
via the variable \textit{f}, the developer also has full
access to the variables passed to \textit{f}, and can make 
arbitrary changes to the arguments in a transparent way, 
i.e. the calling function would never know the arguments 
were modified before being passed to the callee function.

In addition to access to the function and its arguments,
developers can use decorators to insert code immediately before and after 
the function, including leveraging arbitrary arguments (e.g.
\textit{my\_int}) passed to the decorator itself.

The author uses such a decorator to take a “before” and "after" snapshot of the arguments
before and after the function is called.

%
\subsection{Accessing relevant global state}\label{sec:approach-internal-2}

% Next, discuss the need to access global state (read/write).
In addition to the arguments passed directly to the function, any relevant 
global state must also be captured. "Relevant" here refers only to those 
global values read from and/or written to by the function.
The author's code focuses on variables (e.g. the int "c" in the example above), 
disregarding imported modules such as textit{re, os, etc}, detecting such
modules in a seperate parsing step.
% Note that the dis module provides this option via dumping bytecode for f
Access to the global values is non-trival compared to accessing the function 
and its arguments, but still possible.  The first step is to use the \textit{dis}
module to programmatically disassemble the decorating function.  This is only 
required on initial execution of the decoratee as subsequent executions, if any, 
benefit from cached results of the disassembly.
Programmatic disassembly of \textit{add\_ints} function during execution is shown below:

\lstinputlisting[%
  language=TeX,%
  caption={Result of Programmatically Disassembling \textit{add\_ints.py}},%
  label={lst:decorator},%
]{examples/actual\_disassemble\_add\_ints.txt}

Note the LOAD\_GLOBAL command to load the value of 'c'.  Any such global names 
are sanity checked against the \textit{\_\_globals\_\_} attribute of the disassembled function.
%
If the name is found in the \textit{\_\_globals\_\_} dictionary, its name and value is saved for later reference.  Likewise, names and values written to via the STORE\_GLOBAL commands are also parsed, verified to exist in the function's extit{\_\_globals\_\_} attribute, and cached for later use if so.
%
Of note, the disassembly shown above differs from stand-alone disassembly of the same function in the Python interpreter (compare the listing above to the examples/disassemble\_decorator\_with\_decorators.txt file in the accompanying repository.)
This reason for this is that during actual execution the author's code disassembles only the \textit{add\_ints} function, after the decorators have already been unwrapped.
In contrast, disassembling the function in the static, non-executing context of
the Python interpreter reveals the code of the applied decorators.  
%
For the global state read from, those values must be recorded 
in order to monkeypatch them during the unit test programmatically 
created by this work.
%
For relevant global state written to, the decorated function must also record
the new state in order to assert that the state was correctly changed by the function.

\subsection{Detecting exceptions}\label{sec:approach-internal-3}
Detecting exceptions is perhaps the easiest of the three information capture steps.
Any exception can of course be detected with the simple anti-pattern:

\begin{lstlisting}[language=Python]
  try:
    # call the decorated function, e.g.
    f(args, kwargs)
  exception Exception as e:
    # Save the exception type and exception message
  \end{lstlisting}

\subsection{Determing Test Coverage}\label{sec:approach-internal-4}
As the purpose of this work is to advance the science of automated unit test 
creation, a key component is recording test coverage.  The Python \textit{coverage}
module provides support for just this task. However, given the non-standard
approach of this work compared to typical testing, the author programmatically
copied the results from the output of the coverage tool and caches it seperately, 
opting not to use the default coverage database persistantly.

\subsection{Summary of the approach}\label{sec:approach-internal-5}

The author uses all the methods discussed above to take a "before" and “after” 
snapshot of the arguments and relevant global state of each execution. 

(Note that the state of the arguments must also be captured after the function 
executes as called functions may change mutable arguments that persist upon 
return to the caller.) The return value or exception type and exception message
 are also captured, in addition to line test coverage. 
 For each execution of a given function, an instance of the 
 following class is created and the fields populated:

\lstinputlisting[%
  language=Python,%
  numbers=left,
  caption={CoverageInfo: The class that caches all metadata associated with a single execution.},%
  label={lst:CoverageInfo},%
]{examples/coverage_info.py}

In addition, one of each of the FunctionMetaData classes below is populated for each decorated function:

\lstinputlisting[%
  language=Python,%
  numbers=left,
  caption={FunctionMetaData: The class that caches all metadata associated with a single function, to include all associated CoverageInfo classes},%
  label={lst:FunctionMetaData},%
]{examples/function\_metadata.py}

% Discuss overhead and coverage

% Demonstrate how the tests themselves are created with metaprogramming

% vim: spelllang=en\_US

\section{Evaluation}\label{sec:evaluation}

The decorator was evaluated informally by applying it to four test cases
specifically designed to test it.
%

% vim: spelllang=en_GB

\section{Conclusions}\label{sec:conclusions}

At the risk of promoting a new testing paradigm, perhaps 
called Development Driven Testing (DDT), the author 
presented a non-generative-AI approach to 
programmatically generating Python unit tests through
the execution of a single developer (or even AI)-crafted
functional test.

As partly noted above, future work should include: 
\begin{enumerate}
  \item Reduced reliance on \textit{eval}
  \item File and socket I/O
  \item Complete OOP support
  \item Multi-thread and multi-process testing and support
  \item Applicability in other languages with strong metaprogramming support such as C\#, Go, and Zig
\end{enumerate}

The code is available in a single Python file to permit 
easy integration into any supported Python project.
The author wishes to give back to the computing community
and therfore offers this project free and open source
in the hopes that it will be found useful and lay the ground
work for future related research and application.
The author hopes the concepts described herein will be 
applied in other languages with strong metaprogramming support
such as C\#, Go, Zig, etc.  Further information on the code and  documentation, please
find it online at
%
\begin{center}
  \url{REDACTED}
\end{center}


% vim: spelllang=en_GB


\begin{acks}
  A special thanks to my good friend Brad for his insight and expertise with 
  Python testing and the pytest library.
\end{acks}

\bibliographystyle{ACM-Reference-Format}
\bibliography{related}

\end{document}

% vim: spelllang=en_US
