% Add option `kerning` to `microtype`. Since `acmart` already loads `microtype`
% by itself, we need to add this option here to hook into the package-loading
% process.
\PassOptionsToPackage{kerning}{microtype}
%
\documentclass[%
  anonymous=true,%
  authordraft=false,%
  format=sigconf,%
  review=false,%
  screen=true,%
  timestamp=false,%
  pbalance,%
]{acmart}

\copyrightyear{2024}


\usepackage{defs}
\graphicspath{{./}{./img/}}

\newcommand\numModules{\num{118}\xspace}
\newcommand\numIterations{\num[round-precision=2]{30}\xspace}
\newcommand\timeout{\SI{600}{\second}\xspace}
\newcommand\numProjects{\num[round-precision=2]{17}\xspace}
\newcommand\avgCoverage{\SI{67.96650716076552}{\percent}\xspace}
\newcommand\avgCoverageDynaMOSA{\SI{67.96650716076552}{\percent}\xspace}
\newcommand\avgCoverageMIO{\SI{66.99329239952593}{\percent}\xspace}
\newcommand\avgCoverageMOSA{\SI{67.78516671267488}{\percent}\xspace}
\newcommand\avgCoverageRandom{\SI{63.56828612377363}{\percent}\xspace}
\newcommand\avgCoverageWS{\SI{66.9088489056397}{\percent}\xspace}
\newcommand\avgCoverageWSA{\SI{67.53200462581992}{\percent}\xspace}

\setlength {\marginparwidth }{2cm}
\begin{document}

\title{A Python Decorator for Programmatic and Deterministic Unit Test Code Generation}

\author{James Hatfield}
\email{jahatfi@gmail.com}
\orcid{0009-0003-1677-0026}
\affiliation{
  \state{Maryland}
  \country{United States of America}
}


  
  \ccsdesc[300]{Software and its engineering~Software maintenance tools}
  \ccsdesc[500]{Software and its engineering~Software testing and debugging}
  \ccsdesc[500]{Software and its engineering~Empirical software validation}
  \ccsdesc[100]{Software and its engineering~Object oriented frameworks}
  \ccsdesc[100]{Software and its engineering~Object oriented development}

\begin{abstract}
  %
  % Context
  Automated software testing is an area of active research, 
  particularly automated generation of unit tests. 
  %
  % Problem
  Current and previous work such as Generative AI, search-based heuristics, and 
  randomization often led to incomplete coverage and non-deterministic results.
  %
  In addition, some of the previous work lacks a natural way for a human expert 
  to provide input to the automated test generator outside the source code itself. 
  % Insight
  The author presents a Python decorator to generate deterministic Python 
  unit tests from existing higher level tests via metaprogramming.  
  %
  This Python decorator hooks function calls in order to record the 
  metadata (input, output, relevant global variables, and code coverage) 
  of each hooked function. 
  %
  The recorded metadata is then used to automatically generate unit 
  tests for each such function.
  %
  % Contribution
  The author's initial testing confirmed that it can 
  be a force multiplier for the developer, 
  as one passing integration or similar test of the overall 
  system can produce many unit tests.  
  %
  As the initial test is designed by the developer, it provides a natural 
  approach to provide expert human insight that is leveraged to create 
  higher quality unit tests.
  %
  Further, many test cases may be generated for each test.  
  Python programmers can save significant time and effort using the 
  generated test cases as boilerplate if more test cases are required to 
  increase coverage.  
  %

\end{abstract}

\keywords{Python, Automated Test Generation, Metaprogramming}

\maketitle

\section{Introduction}\label{sec:introduction}
Advocates of Test Driven Development (TDD) such a Harry Percival,
the author of Test-Driven Development with Python,
argue that software developers should follow the TDD process in
which they develop software by first writing tests,
then the bare minimum amount of code to make those tests pass,
before beginning the cycle anew with more tests \cite{percival2014test} page 3.
%
He advocates both functional and unit tests, making the following distinction:
“functional tests test the application from the outside, from the point of view
of the user.
%
Unit tests test the application from the inside,
from the point of view of the programmer” (Percival, 22)
%
Much later in the book Percival defines an “integration test” as
one that depends on and interacts with some external system (Percival 470.)
%
“Integration test” is sometimes use in common vernacular in the same way that
Percival uses “functional test”.
%
This paper will do the same.

In the author’s experience, immature organizations (or solo developers
on small or personal projects) often don’t adhere to the TDD approach,
and end up writing the tests after the development of the code or
fail to write tests at all.
%
Alternatively, some teams may focus primarily on function or integration tests
and opt not to develop unit tests, despite the pros (not to mention cons) of
each.
%
This is exemplified in this StackExchange QA: https://softwareengineering.
%
stackexchange.com/questions/204786/do-i-need-unit-test-if-i-already-have-integration-test.
%
The main pros and cons for functional versus unit tests are summarized very
briefly in the short table below:
%
Since unit tests are tedious and time consuming to develop, the body of academic
work on generating them automatically is extensive, starting with classic
algorithms like search and randomization based \cite{Luk22Pynguin0170} and
rapidly expanding with the advent of generative AI 
\cite{bhatia2023unit,takerngsaksiri2024tdd,wang2024software}.
%
These AI methods generate tests based on a variety of inputs, typically the code
itself and some other input such as human prompts \cite{lahiri2022interactive},
natural language requirements \cite{wang2024software}, or entire projects
\cite{rao2023cat}.

In the current research this author did not discover a deterministic (i.e.
non-generative AI) unit test generation paradigm that monitored live execution
of the code.
%
Thus, this work is perhaps most similar to lahiri2022interactive, though
differs in that rather than prompt users interactively for sample inputs and
outputs, it automatically detects such inputs, outputs, and changes in global
state by monitoring actual execution of the code.
%
The decorator is written in Python and requires at least Python 3.9 to run.
%
The decorator was developed and tested on a Windows 10 PC.  The author 
did not test it on *nix environments but given the cross-platform support of
Python, the author believes it should run in *nix environments with no more than
minor modifications.
%
The author releases it under the GNU LGPL open-source licence.
% vim: spelling=en_US
\section{Recording the Execution of a Python Function}\label{sec:approach}
% Overview of concepts
The author set out to record the execution of Python functions
in such a way as to enable exact reproduction of that 
function call at at later time.  The following components
must be recorded in some fashion in order to do so:
\begin{enumerate}
  \item The function itself
  \item The arguments to the function, including kwargs
  \item Any relevant global state (e.g. variables)
  \item Exceptions raised
  \item Test coverage
  \item Files/databases read from and/or written to
  \item Data sent/received via a socket
\end{enumerate}

The last two are left for future work, but this paper demonstrates how to use 
the initial components. The subsections that follow discuss exactly how to 
access or determine this info and cache for subsequent creation of unit tests
with this information.
%%
%% General information about The decorator
%%
\subsection{Accessing the function and its arguments}\label{sec:approach-internal-1}

Python enables trivial access to a function and 
its arguments by another function via the concept of 
decorator functions.

Not to be confused with the decorator pattern, a Python decorator is simply a 
function that calls another, thereby permitting the developer to place new code 
before and/or after calling the original “decorated” function.  
The decorator function has full access to both the decorated 
(or "wrapped" function) \textit{f} as well as all the 
arguments passed to \textit{f}, both \textit{args} and \textit{kwargs}.  In other 
languages this kind of wrapping is often referred to as 
“function hooking” or "function call interception" 
 \cite{kang2018function}. Any number of decorators can be applied to a function 
in Python, creating a figurative Russian nesting doll of
functions calling functions, each with the ability to access the 
arguments of the function it wraps and modify the 
return value before return. A Python decorator is applied with the \lq@\rq 
symbol as shown below \footnote{The author optimized this and other code snippets for display in 
this paper; therefore they may differ slightly from their original 
source found in the associated repository}:

\lstinputlisting[%
  language=Python,%
  numbers=left,
  caption={decorator.py: A sample decorator that takes one argument.},%
  label={lst:decorator},%
]{examples/decorator.py}

Running the code above yields the output below. Note the apparently erroneous
output, as the \textit{add\_ints} function is unaware that its first argument was
modified by the \text{inner\_most\_decorator}.

\lstinputlisting[%
  language=bash,%
  numbers=left,
  caption={Output of decorator.py},%
  label={lst:decorator},%
]{examples/decorator\_result.txt}

As shown in Listing 2, not only can the Python developer access the function
via the variable \textit{f} (e.g. line 14), the developer also has full
access to the variables passed to \textit{f}, and can make 
arbitrary changes to the arguments in a transparent way (line 11), 
i.e. the calling function would never know the arguments 
were modified before being passed to the callee function.

In addition to access to the function and its arguments,
developers can use decorators to insert code immediately before and after 
the function (lines 8-13, 15, respectively), including leveraging arbitrary
arguments (e.g. \textit{my\_int}) passed to the decorator itself.

The author uses such a decorator (specifically named 
\break
\textit{unit\_test\_generator\_decorator} in the referenced repository) to take
 a “before” and "after" snapshot of the arguments
before and after the function is called.

%
\subsection{Accessing relevant global state}\label{sec:approach-internal-2}

% Next, discuss the need to access global state (read/write).
In addition to the arguments passed directly to the function, any relevant 
global state must also be captured. "Relevant" here refers only to those 
global values read from and/or written to by the function.
The author's code focuses on variables (e.g. the int \textit{c} in the example above), 
disregarding imported modules such as \textit{re, os, etc}, detecting such
modules in a separate parsing step.
% Note that the dis module provides this option via dumping bytecode for f
Access to the global values is non-trival compared to accessing the function 
and its arguments, but still possible.  The first step is to use the \textit{dis}
module to programmatically disassemble the decorating function.  This is only 
required on initial execution of the decoratee as subsequent executions, if any, 
benefit from cached results of the disassembly.
Programmatic disassembly of \textit{add\_ints} function during execution is shown below:

\lstinputlisting[%
  language=TeX,%
  numbers=left,
  caption={Result of Programmatically Disassembling \textit{add\_ints.py}},%
  label={lst:decorator},%
]{examples/actual\_disassemble\_add\_ints.txt}

Note the LOAD\_GLOBAL command to load the value of \textit{c}.  Any such global names 
are sanity checked against the \textit{\_\_globals\_\_} attribute of the 
disassembled function.
%
If the name is found in the \textit{\_\_globals\_\_} dictionary, its name and 
value is saved for later reference.  Likewise, names and values written to via 
the STORE\_GLOBAL commands are also parsed, verified to exist in the function's
 \textit{\_\_globals\_\_} attribute, and cached for later use if so.
%
Of note, the disassembly shown above differs from stand-alone disassembly of 
the same function in the Python interpreter (compare the listing above to the
 examples/disassemble\_decorator\_with\_decorators.txt file in the 
 accompanying repository-it's 80 lines long.)
They differ as during actual execution the author's code
 disassembles only the \textit{add\_ints} function, after the decorators
  have already executed, leaving only the original  \textit{add\_ints}
  function for disassembly.
In contrast, disassembling the function in the static, non-executing context of
the Python interpreter reveals the code of all applied decorators.  
%
For the global state read from, those values must be recorded 
in order to monkeypatch them during the unit test programmatically 
created by this work.
%
For relevant global state written to, the decorated function must also record
the new state in order to assert that the state was correctly changed by the function.

\subsection{Detecting exceptions}\label{sec:approach-internal-3}
Detecting exceptions is perhaps the easiest of the three information capture steps.
Any exception can of course be detected with the simple anti-pattern:

\begin{lstlisting}[language=Python, caption={Catching and recording exceptions}]
  try:
    # call the decorated function, e.g.
    f(args, kwargs)
  exception Exception as e:
    # Save the exception type and exception message
  \end{lstlisting}

\subsection{Determing Test Coverage}\label{sec:approach-internal-4}
As the purpose of this work is to advance the science of automated unit test 
creation, a key component is recording test coverage.  The Python \textit{coverage}
module provides support for just this task like so:

\lstinputlisting[%
  language=Python,%
  numbers=left,
  caption={Coverage: Calling a function and capturing the test coverage.},%
  label={lst:Coverage},%
]{examples/coverage.py}

However, given the non-standard
approach of this work compared to typical testing, the author programmatically
copied the results from the output of the coverage tool and cached it separately, 
opting not to use the default coverage database persistantly.

\subsection{Summary of the approach}\label{sec:approach-internal-5}

The author uses all the methods discussed above to take a "before" and “after” 
snapshot of the arguments and relevant global state of each execution. 

(Note that the state of the arguments must also be captured after the function 
executes as called functions may change mutable arguments that persist upon 
return to the caller.) The return value or exception type and exception message
 are also captured, in addition to line test coverage. 
 For each execution of a given function, an instance of the 
 following class is created and the fields populated:

\lstinputlisting[%
  language=Python,%
  numbers=left,
  caption={CoverageInfo: The class that caches all metadata associated with a single execution.},%
  label={lst:CoverageInfo},%
]{examples/coverage_info.py}
In addition, one of each of the FunctionMetaData classes below is populated for each decorated function:

\lstinputlisting[%
  language=Python,%
  numbers=left,
  caption={FunctionMetaData: The class that caches all metadata associated with
   a single function, to include all associated CoverageInfo classes},%
  label={lst:FunctionMetaData},%
]{examples/function\_metadata.py}

\section{Programmatically Tuning the Decorator}\label{sec:decorator tuning}

\subsection{Selectively bypassing the decorator}\label{sec:tuning-1}

As the reader may imagine, the decorator described above adds significant overhead 
the execution time of the overall test, as the function call is intercepted,
arguments inspected, copied etc.  To mitigate this overhead the author provides 
a variety of methods to 
reduce this overhead by "bypassing" the decorator, i.e. calling the decorated 
function and simply returning the result without further action.

The first and perhaps most obvious approach is to set a coverage threshold such 
that the decorator is effectively disabled once a desired level of coverage 
(e.g. 80\%) is achieved. 

As a corollary, one could also define a specific number 
of executions to capture, after which the decorator would be disabled. 
The author supports either option by setting either the \textit{percent\_coverage} or 
\textit{sample\_count} option to the decorator to the desired value.  Setting 
\textit{percent\_coverage} to a value greater than 100 will capture all 
executions of the decorated function.
\lstinputlisting[%
  language=Python,%
  numbers=left,
  caption={Programmatically "bypassing" the decorator by immediately returning 
  the function results when desired coverage met},%
  label={lst:ProgrammaticallyTuning},%
]{examples/programmatictuning.py}

The careful reader may wonder if the \textit{sample\_count} option may result in 
duplicate tests cases, e.g. five test cases are captured but three are duplicates, 
yielding only three unique test cases.  This would in fact be the case, however, the author 
addressed this by hashing the inputs (global values, args and kwargs) and 
caching the hash in a set name \textit{hashed\_inputs}.  If a function has
already been called with the exact same inputs, the decorator immediately returns
the result in the same fashion as shown above, as the metadata associated with
 executing the decorated function
with those inputs was already captured and recorded.  Thus, all test cases 
generated are unique.  This represents the last mitigation
currently in the code.  In future work the author hopes to return the cached 
result/raise the same exception of this execution to save even more overhead.  

It is also important to note that the decorator need only be applied until the 
test cases are generated and the developer is satisfied that the created unit
tests will suffice.  At that point the decorator can be completely removed
from the functional or integration test, thereby restoring the runtime of
the original test back to its original, faster runtime.  The author 
was able to generate successful unit tests for a C function parser, patch two
bugs revealed by the unit tests, re-run the function test, then remove 
the decorator once confident the bugs were fixed.  The author only had to run 
the slower, decorated integration test twice before removing the decorator.
At that point he was confident that the generated unit test and dozens of 
tests cases created from the function test sufficed for unit testing apart 
from the functional test.

\subsection{Selectively keeping specific test cases}\label{sec:tuning-1}
As the same function may be executed multiple times during an ad hoc or 
function test, the user may opt not to record every execution by determining
if the current execution didn’t add any new coverage. There are a few 
ways to do this; a few are explored below. The reader should bear in mind
that fully orthogonal test cases, as measured by coverage, are (1 not feasible 
(all test cases will cover the initial lines until the first conditional statement) 
(2 redundant tests can still hold value and (3 some code such as function calls 
and regular expressions may have one line in source code, but multiple branches in the 
underlying library or machine code.  Nevertheless, the value added by redundant tests
likely follows the Law of Diminishing Returns, hence the following sections
on reducing such redundancy.
 
 \subsubsection{Solving the Minimum Set Cover Problem}\label{sec:tuning-2}
 \hfill\\
If the developer caches all records initially then the Minimum Set Cover Problem
\cite{hassin2005better} is the exact algorithm required to select the minimum test cases.
This classic NP-Hard computer science problem involves finding the smallest number
  of sets whose union is equal to or greater than (“covers”) some universe $U$ of elements.  
  In this context, the coverage (set of line number executed) of each test is one set $S_i$,
 and the set of all line numbers of a function is the universe $U_f$ to cover.  
 As noted, this requires caching the unique coverage of each test, which may 
 be prohibitive from a memory standpoint, though it may reduce the runtime 
 compared to the alternate approaches described below, as the pruning 
 only takes place after the tests are complete, as opposed to making 
 pruning decisions during each test.

\subsubsection{Check for new coverage via unified set}\label{sec:tuning-2}
\hfill\\
Another approach is to maintain a unified set $U_t$ that is the union of all lines 
covered by all recorded tests from $S_0$ to $S_n$: 

\begin{equation*}
  U_t = S_0 \cup S_1 ...\cup... S_n
\end{equation*}
then only keeping the current record $S_i$ if any currently 
covered lines are not in the unified set:
\begin{equation*}
  S_i \not \subset U_t
\end{equation*}
then updating the unified set with the new coverage information like so:
\begin{equation*}
  U_t \Leftarrow U_t + S_i
\end{equation*} 
With this approach the individual coverage information $S_i$ is not maintained 
($U_t$ is updated, then $S_i$ is discarded), making it impossible to drop superseded 
coverage records over time.  100\% coverage would mean that:
\begin{equation*}
  U_f = U_t
\end{equation*} 
This approach is the simplest of the three described here, but 
\subsubsection{Individual Subset Approach}\label{sec:tuning-2}
\hfill\\
Alternatively, one can maintain coverage information for all previous
 executions individually and check the current set of covered lines
  against all previous coverage sets.  This would permit deletion of 
  previous records that covered a subset of the current execution.  

This is more effort than the previous approach but retains fewer records 
with the same total coverage.  A simple sample of the latter approach is shown here:

\vskip .2cm
\begin{tabular}{|>{\centering\arraybackslash}m{1.6cm}|>{\centering\arraybackslash}m{1.25cm}|>{\centering\arraybackslash}m{4.5cm}|}
    \hline
    \multirow{2}{*}{\textbf{Execution \#}} & \textbf{Coverage (line \#s)} & \textbf{Action} \\
    \hline
    1 & 1-5 & Keep this record, since it is the first one.\\
    \hline
    2 & 1-7 & Keep this and drop record \#1, as that coverage is only a subset of this coverage.\\
    \hline
    3 & 1-2 & Don’t keep this record; it is a subset of an existing record (\#2)\\
    \hline
\end{tabular}
\captionof{table}{Hypothetical scenario \#1 demonstrating record pruning by checking for subsets}
\vskip .2cm
Unfortunately, this subset approach still leaves room for redundant tests.  
Consider a new scenario:
\vskip .2cm
\begin{tabular}{|>{\centering\arraybackslash}m{1.6cm}|>{\centering\arraybackslash}m{1.25cm}|>{\centering\arraybackslash}m{4.5cm}|}
    \hline
    \multirow{2}{*}{\textbf{Execution \#}} & \textbf{Coverage (line \#s)} & \textbf{Action} \\
    \hline
    1 & 1-5 & Keep this record, since it is the first one.\\
    \hline
    2 & 6-10 & Keep this one, it covers new code.\\
    \hline
    3 & 5-6 & Keep this record, it’s not a subset of any previous record.\\
    \hline
\end{tabular}
\captionof{table}{Hypothetical scenario \#2 demonstrating record pruning by checking for subsets}
\vskip .2cm

Since the subset approach does not aggregate all  the coverage into a 
single unified set, the record for execution \#3 would be kept, but this would 
be redundant, as executions 1 and 2 already executed lines 5 and 6, respectively.

\subsubsection{Selected Approach}\label{sec:tuning-2}

The author implemented the last approach such that setting the \textit{keep\_subsets}
options to True would not drop redundant records.  The author did not implement 
the Minimum Set Cover problem or its weighted variants, but encourages 
the interested reader to do so.  The author records the execution time of each 
execution as the \lq cost \rq. This value could be applied in a weighted 
variant of the Minimum Set Cover problem, i.e. select the "best" test cases
such that coverage is maximized and run time is minimized.
   
\section{Generating the Unit Tests via Metaprogramming}\label{sec:generating-tests}

With all the metadata regarding the functions and their executions recorded, 
the next step was to programmatically generate the unit tests with all their 
captured test cases and place them in a file named 
\textit{test\_FUNCTION\_NAME.py}. 
A simple call to \textit{generate\_all\_tests\_and\_metadata} at the end of 
the functional test will do just that.
 Calling this function eventually executes 
\textit{auto\_generate\_tests} that uses metaprogramming to:
\begin{enumerate}[leftmargin=*]
  \item Build a list of imports
  \item Define global variables constant across all test cases
  \item Convert values to valid (i.e. canonical) Python code
  \item Build a comment indicate the level of coverage achieved
  \item Collect parameters to create a parameterized pytest test
  \item Monkeypatch all non-constant relevant globals read from
  \item Assert result is correct
  \item Assert modified globals are correct
  \item Assert expected exceptions and their messages are correct
  \item Write all of the above to a syntactically correct Python file
  \item Use subprocess to lint and format the result with \textit{black} and \textit{ruff}
\end{enumerate}

To aid the developer's troubleshooting efforts, the contents of each FunctionMetaData
class are also dumped to their own .json file, one per decorated function (hence
the \textit{and\_metadata} suffix noted above).  This file permits the 
developer to easily inspect the inputs, outputs, and other data
associated with the recording without the distraction of the test code.

Thus, if the function \textit{foo} is decorated with \textit{unit\_test\_decorator},
the generated unit test code can be found in \textit{test\_foo.py}, and the 
metadata can be found in a JSON file name named \textit{foo.json}
% Discuss overhead and coverage

% Demonstrate how the tests themselves are created with metaprogramming

% vim: spelllang=en\_US

\section{Tester, Test Thyself}\label{sec:evaluation}

The decorator was evaluated informally with three separate approaches.
The author began by crafting simple but demonstrative test cases 
(found in the \textit{tests/} folder in the associated repository.) Next,
the author applied the concept of programmatically generated unit tests
to the code itself with positive results.  Finally, as previously mentioned, 
the author applied his work to another project involving parsing C code. 
These three approaches are described detail below:

\subsection{Manually Created Tests}\label{sec:eval-1}
The author created a variety of tests to ensure the unit test generation 
code functioned properly. The Procedural Division example is explained in depth
below and the others are briefly summarized.

The \textit{tests/example\_procedural\_division} test was intended 
to test procedural (as opposed to object-oriented or functional) code that:  

\begin{enumerate}
    \item Returned a string given two ints
    \item Wrote to a global variable
    \item Raised two different types of exceptions
\end{enumerate}

The author wrote a \textit{divide\_ints} function as follows:

\lstinputlisting[%
  language=Python,%
  numbers=left,
  caption={divide\_ints.py},%
  label={lst:Divide Ints},%
]{examples/divide\_ints\_1.py}

Then wrote an ad-hoc "test" without assertions, but calls 
\textit{divide\_ints} with a variety of inputs, as shown below:

\lstinputlisting[%
  language=Python,%
  numbers=left,
  caption={test divide\_ints()},%
  label={lst:Test Divide Ints},%
]{examples/divide\_ints\_2.py}

Running this code like so:

\begin{lstlisting}[language=bash, caption={Executing example to create unit test}]
    $ python divide_ints.py
\end{lstlisting}
    
Produces the following \textit{test\_divide\_ints.py} containing the following 
unit test (modified only slightly below to reduce line breaks):

\lstinputlisting[%
  language=Python,%
  numbers=left,
  caption={test divide\_ints()},%
  label={lst:Test Divide Ints},%
]{examples/divide\_ints\_3.py}

Lines 18-20 set up the parameterization decorator, defining the inputs to each test,
and the inputs to each are defined in a list of tuples spanning lines 23-33, 
34-44, 45-57.  The actual test function follows starting on line 60, monkeypatches
the required ERROR\_CODE global variable on lines 75-76, tests for expected 
exceptions on lines 77-80, and only calls the function on line 82 if it expects 
no exception. The result is verified on lines 82-83
\footnote{due to the complexity of converting from Python object to string
(and sometimes back again), the author was forced due to time contraints to
rely on the \textit{eval} function to convert strings to valid Python.  
The author acknowleges this is a bad practice and hopes to fix it in
future revisions.} An unexpected exception (or incorrect 
(exception, exception message) 
pair) would cause the unit test to fail.  Finally, global variables expected to be modified 
are checked for correct values on lines 85-92.

Executing the unit test is simple\footnote{Due to the verbose way pytest prints the 
all the parameters of  parameterized tests such as these, the author removed
them for the sake simpler display.  The reader is encouraged to run the code 
as shown on their own machine}:

\begin{lstlisting}[language=bash, numbers=left, caption={Running one of the generated unit tests}]
pytest -s -v test_divide_ints.py
========== test session starts ==========
platform win32 -- Python 3.11.7, pytest-7.4.4, pluggy-1.4.0 -- <@\textcolor{orange}{PATH REDACTED}@>\.venv\Scripts\python.exe
cachedir: .pytest_cache
rootdir: <@\textcolor{orange}{PATH REDACTED}@>\tests\example_procedural_division
plugins: cov-5.0.0
collected 4 items

test_divide_ints.py::test_divide_ints[<@\textcolor{orange}{Test \#1 arguments SNIPPED}@>] <@\textcolor{green}{PASSED}@>
test_divide_ints.py::test_divide_ints[<@\textcolor{orange}{Test \#2 arguments SNIPPED}@>] <@\textcolor{green}{PASSED}@>
test_divide_ints.py::test_divide_ints[<@\textcolor{orange}{Test \#3 arguments SNIPPED}@>] <@\textcolor{green}{PASSED}@>
test_divide_ints.py::test_divide_ints[<@\textcolor{orange}{Test \#4 arguments SNIPPED}@>] <@\textcolor{green}{PASSED}@>
\end{lstlisting}

The reader is encouraged to study the other examples in the repository.
All examples can be executed via the 
\textit{tests/test\_all.*} scripts.  Those
scripts can easily be used to determine the invocation required to run each 
test individually.  They are summarized very briefly below:

The \textit{tests/example\_fizzbuzz} test is nearly identical to the 
test described above: tests modification of a global value.

The \textit{tests/example\_all\_types} test was designed to test
the code against a variety of built-in Python types, 
such as ints, strings, sets, lists, tuples, and dictionaries.

The \textit{tests/example\_pass\_by\_assignment} test was designed to test
functions that modify mutable arguments such as lists. Not a comprehensive test,
but demonstrates that the concept works on one such example.

The \textit{tests/example\_oo\_car} test was designed to test
object-oriened code. Not complete at the time of submission, but worked in a 
previous branch prior to adding support for the test above.
The author is confident that this feature will be working by 
fall of 2024.
%
\subsection{Application of this code to itself}\label{sec:eval-2}

The author naturally sought to determine the effectiveness of automatic
test code generation against the very code doing just that.  Although not 
every function in the referenced code was decorated for automatic testing, 
as of this writing five functions in 
\textit{unit\_test\_generator\_decorator.py} were successfully decorated, 
resulting in automatically generated, successfully passing unit tests for this
very code.  Using the previous Procedural Division example (and others) 
the reader should see the following test files created in addition to 
\textit{test\_divide\_ints.py}

\begin{enumerate}
    \item test\_coverage\_str\_helper.py
    \item test\_meta\_program\_function\_call.py
    \item test\_normalize\_args.py
    \item test\_update\_global.py
    \item test\_update\_metadata.py
\end{enumerate}

These tests also pass when run with this command in 
\linebreak
\textit{tests/example\_procedural\_division/} directory:
\begin{lstlisting}[language=bash, caption={Running all generated unit
   tests for the division example}]
    pytest -s -v .
\end{lstlisting}

\subsection{Application of this code to an external project}\label{sec:eval-2}
As previously mentioned and put simply, this project was applied to a C code parser, 
specifically to a Python function designed to parse C code line by line
and extract the name of any functions on each line.  The author applied this
decorator to the parsing function and passed a C file to the parser function.
%
Due to the design of the parser, some lines were parsed multiple times.  
However, the decorator prevented duplicate results as previously discussed, 
early instantly producing over three dozen unique test cases 
from a single functional test and revealing two bugs to the author.
%
As noted, some work remains, such as completing the decorator and helper logic
to handle object-oriented code.  However, due to the variety of successful 
tests noted above, the author maintains an optimistic outlook that this concept
of generating unit tests from functional tests can be a valuable paradigm.

% vim: spelllang=en_GB

\section{Conclusions}\label{sec:conclusions}

The increasing popularity of Python
requires the availability of many types of tools
to aid the developers.
%
We introduce \pynguin,
an automated unit test generation framework,
designed to support developers
when implementing unit tests manually.
%
\Pynguin is available as a command-line application,
which is the de-facto standard for many developer-aiding tools.
%
While this provides great flexibility for users,
\pynguin's modular design also enables further extensions
and research in the field of automated unit test generation
for dynamically typed languages.
%

\lstinputlisting[%
  float=t,%
  language=Python,%
  caption={An excerpt of the test cases generated by \pynguin.},%
  label={lst:test-triangle},%
]{examples/test_triangle.py}
%
In this work we have summarised the features of \pynguin.
%
By providing \pynguin as open source,
we hope to foster further research
as well as its applicability in practice.
%
Further information on \pynguin,
its documentation,
and source code
are available at
%
\begin{center}
  \url{https://www.pynguin.eu}
\end{center}


% vim: spelllang=en_GB


\begin{acks}
  A special thanks to my good friend Brad for his insight and expertise with 
  Python testing and the pytest library.
\end{acks}

\bibliographystyle{ACM-Reference-Format}
\bibliography{related}

\end{document}

% vim: spelllang=en_US
