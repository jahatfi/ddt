\documentclass[%
  tikz,%
  %margin=3mm,%
]{standalone}

\usepackage[T1]{fontenc}
\usepackage[utf8]{inputenc}
\usepackage{libertine}
\usepackage{graphicx}

\usepackage{pgf}

\usetikzlibrary{%
  arrows,%
  arrows.meta,%
  backgrounds,%
  calc,%
  positioning,%
  shapes.symbols,%
}

\makeatletter
\pgfdeclareshape{umlnote}{%
  \inheritsavedanchors[from=rectangle]%
  \inheritanchorborder[from=rectangle]%
  \inheritanchor[from=rectangle]{center}%
  \inheritanchor[from=rectangle]{north}%
  \inheritanchor[from=rectangle]{north east}%
  \inheritanchor[from=rectangle]{north west}%
  \inheritanchor[from=rectangle]{south}%
  \inheritanchor[from=rectangle]{south east}%
  \inheritanchor[from=rectangle]{south west}%
  \inheritanchor[from=rectangle]{west}%
  \inheritanchor[from=rectangle]{east}%
  \backgroundpath{%
    \southwest \pgf@xa=\pgf@x \pgf@ya=\pgf@y%
    \northeast \pgf@xb=\pgf@x \pgf@yb=\pgf@y%
    \pgf@xc=\pgf@xb \advance\pgf@xc by-14pt%
    \pgf@yc=\pgf@yb \advance\pgf@yc by-14pt%
    \pgfpathmoveto{\pgfpoint{\pgf@xa}{\pgf@ya}}%
    \pgfpathlineto{\pgfpoint{\pgf@xa}{\pgf@yb}}%
    \pgfpathlineto{\pgfpoint{\pgf@xc}{\pgf@yb}}%
    \pgfpathlineto{\pgfpoint{\pgf@xb}{\pgf@yc}}%
    \pgfpathlineto{\pgfpoint{\pgf@xb}{\pgf@ya}}%
    \pgfpathclose%
    \pgfpathmoveto{\pgfpoint{\pgf@xc}{\pgf@yb}}%
    \pgfpathlineto{\pgfpoint{\pgf@xc}{\pgf@yc}}%
    \pgfpathlineto{\pgfpoint{\pgf@xb}{\pgf@yc}}%
    \pgfpathlineto{\pgfpoint{\pgf@xc}{\pgf@yc}}%
  }%
}

% Key to add font macros to the current font
\tikzset{%
  add font/.code={%
    \expandafter\def\expandafter\tikz@textfont\expandafter{\tikz@textfont#1}%
  }%
}

% Define default style for this node
\tikzset{umlnote/port labels/.style={font=\sffamily}}
\tikzset{every umlnote node/.style={%
    draw, inner sep=2mm, outer sep=0pt, font=\normalsize,%
    minimum height=2.1cm%
}}
\makeatother

\tikzset{%
  gridbackground/.style={%
    background rectangle/.style={fill=white},%
    background grid/.style={draw=black!10, step=5mm, ultra thin},%
    show background rectangle,%
    show background grid,%
  }%
}

\newcommand*\circled[1]{%
  \tikz[baseline=(char.base)]{%
    \node[%
      shape=circle,%
      draw,%
      inner sep=2pt,%
      minimum size=0pt,%
      fill=black!20,%
    ] (char) {#1};%
  }%
}

\begin{document}

\begin{tikzpicture}[%
  %gridbackground,%
  every node/.style={%
    align=center,%
    minimum height=1.0cm,%
    minimum width=2cm,%
  },%
]

%% === Begin Nodes =============================================================
\node [cloud,
    minimum width=1.8cm,
    text width=2cm,%
    aspect=3.0,
    draw=black] (c) at (0,1.6) {Context};
\node at (1.0,2.1) {\circled{4}};


\node [umlnote,text width=2cm] at (0, -1.4) (file) {Python \\ Module};
\node at (1.0,-0.9) {\circled{1}};

\node [rectangle,draw=black,text width=2cm] at (4.0, 0) (analysis) {Analysis};
\node at (5.0,0.5) {\circled{2}};

\node [rectangle,draw=black,text width=2cm] at (7.5,0) (cluster) {Test \\ Cluster};
\node at (8.5,0.5) {\circled{3}};

\node [rectangle,draw=black,text width=2cm] at (11.0, 0) (gen) {Test Case \\ Generation};
\node at (12.0,0.5) {\circled{5}};

\node [rectangle,draw=black,text width=2cm] at (11.0, 1.6) (exec) {Test Case \\ Execution};
\node at (12.0,2.1) {\circled{6}};

\node [rectangle,draw=black,text width=2cm] at (6.0, -1.6) (dynamosa) {DynaMOSA};

\node [rectangle,draw=black,text width=2cm] at (8.5, -1.6) (ws) {Whole Suite};

\node [rectangle,draw=black,text width=2cm] at (11.0, -1.6) (ellipsis) {\dots};

\node [rectangle,draw=black,text width=2cm] at (14.5, 0) (assertion) {Assertion \\ Generation};
\node at (15.5,0.5) {\circled{7}};

\node [rectangle,draw=black,text width=2cm] at (14.5, -1.6) (mutation) {Mutation \\ Engine};
\node at (15.5,-1.1) {\circled{8}};

\node [rectangle,draw=black,text width=2cm] at (18.0, 0) (export) {Test Case \\ Export};
\node at (19.0,0.5) {\circled{9}};

\node [umlnote,text width=2cm] at (21.5,0) (tc) {Python Test \\ Module};
\node at (22.7,0.5) {\circled{10}};
%% === End Nodes ===============================================================

%% === Begin Arrows ============================================================
\draw [-{Latex[]}, rounded corners=6pt] (file) -- (0,0) -- (analysis);

\draw [-{Latex[]}, rounded corners=6pt] (c) -- (4.0,1.6) -- (analysis);

\draw [dashed,->]
    (file) --
    node[xshift=-0.6cm,yshift=0.3cm] {\guillemotleft uses\guillemotright} ++(c);

\draw [-{Latex[]}] (analysis) -- (cluster);

\draw [-{Latex[]}] (cluster) -- (gen);

\draw [-{Latex[]}] (gen) to [out=170, in=180] (exec);

\draw [-{Latex[]}] (exec) to [out=0, in=10] (gen);

\draw [-{Triangle[open]}, rounded corners=6pt]
    (dynamosa) -- (6.0, -0.8) -- (11.0, -0.8) -- (gen);

\draw [-{Triangle[open]}, rounded corners=6pt]
    (ws) -- (8.5, -0.8) -- (11.0, -0.8) -- (gen);

\draw [-{Triangle[open]}] (ellipsis) -- (gen);

\draw [-{Latex[]}] (gen) -- (assertion);

\draw [-{Latex[]}] (assertion) to [out=350, in=0] (mutation);

\draw [-{Latex[]}] (mutation) to [out=180, in=190] (assertion);

\draw [-{Latex[]}] (assertion) -- (export);

\draw [-{Latex[]}] (export) -- (tc);
%% === End Arrows ==============================================================

%% === Begin Bounding Box ======================================================

\draw (2.0, 2.6) -- (2.0, -2.5) -- (19.5, -2.5) -- (19.5, 2.6) -- (2.0, 2.6);

%% === End Bounding Box ========================================================

%% === Put Pynguin Logo in right top corner of bounding box ====================

\node [inner sep=0pt] (pynguin) at (17.5, 1.5)
    {\includegraphics[width=3cm]{pynguin-logo}};

%% === End Put Pynguin Logo in right top corner of bounding box ================


\end{tikzpicture}

\end{document}
